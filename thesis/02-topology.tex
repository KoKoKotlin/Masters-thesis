\section{Topology, Topological Algebra and Algebraic Topology}
\subsection{Topology}

\subsubsection{Topological Spaces}
\begin{defin}
  Let $X$ be a set. A \textbf{topology} $\T$ on $X$ is a collection of subsets of $X$ obeying the follwing axioms
  \begin{enumerate}
    \item $X \in \T \ni \emptyset$,
    \item $\bigcup\limits_{\alpha \in I}A_{\alpha} \in \T$, where $I$ is an abitrary index set and $A_{\alpha} \in \T$ for all $\alpha \in I$,
    \item $\bigcap\limits_{i=0}^n A_i \in \T$ for $n \in \N$ and $A_i \in \T$.
  \end{enumerate}
  
  The tuple $(X, \T)$ where $\T$ is a topology on $X$, is called a \textbf{topological space}.
\end{defin}

In the following we will say that $X$ is a topological space without mentioning the explicit topology every time. 

\begin{defin}
  Let $X$ be a topological space. A family of open sets $\mathcal{B} \subseteq \mathcal{T}_X$ is called a \textbf{basis} of the topology on $X$ if
  \begin{equation*}
    \forall U \in \mathcal{T}_X\forall x \in U\exists B \in\mathcal{B}\colon x \in B \subseteq U.
  \end{equation*}
  A family of open sets $\mathcal{S} \subseteq \mathcal{T}_X$ is called a \textbf{subbasis} of the topology on $X$ if
  \begin{equation*}
    \left\{\bigcap \mathcal{S}'\colon \mathcal{S}'\in \PowSF(\mathcal{S})\right\}
  \end{equation*}
  is a basis of the topology on $X$.
\end{defin}
The intuitive idea of the definition above is that every open set in the topological space $X$ can be represeted by an arbitrary union of basis elements. Additionally each basis element is only a finite intersection of subbasis elements.

In many cases it is not possible to describe the topology on $X$ directly and one can only give a basis for the topology. Then one can talk about $\mathcal{T}(\mathcal{B})$ which is the topology generated by the set $\mathcal{B}$. This topology consists of all arbitrary unions of elements in $\mathcal{B}$. The next Lemma shows that not every subset of $\PowS(X)$ is a basis for a well-defined topology topology on $X$.
\begin{lemma}
  Let $X$ be a set and let $\mathcal{B} \subseteq \PowS(X)$, then the following are equivalent:
  \begin{itemize}
  \item $\mathcal{B}$ is the basis of a topology on $X$,
    \item $\bigcup \mathcal{B} = X$ and $\forall B_1, B_2 \in \mathcal{B}\exists B_3 \in \mathcal{B}\colon \forall x \in B_1\cap B_2\colon x \in B_3 \subseteq B_1\cap B_2$.
  \end{itemize}
\end{lemma}

\begin{proof}
  See \cite[p. 78f.]{MunTop}.
\end{proof}

\begin{ex}\label{ex:top}
  Here are some examples of topological spaces which will be needed later.
\begin{enumerate}[label=\roman*.)]
  \item \label{ex:top-chaotic} Let $X$ be a nonempty set. Then the set $\T_X := \{X, \emptyset\}$ is a topology on $X$ called the \textbf{chaotic topology}.
  \item \label{ex:top-order} Let $(P, <)$ be a totally ordered set\footnote{see \cite[p. 24]{MunTop} for definition of total order (there simple order)}. Define for $a, b \in P$ with $a < b$ the \textbf{open interval} from a to b as $(a,b) := \{ x \in X\colon a < x \: \land \: x < b \}$ and similiarly define the half open intervals $[a,b) := \{a\} \cup (a,b)$ and $(a,b] := (a,b) \cup \{ b \}$. 
  Then the \textbf{order topology} on $P$ is generated by the basis
  \begin{equation*}
    \mathcal{B} := \{ (a,b) \colon a, b \in P \} \: \cup \: \{[a_0, b)\colon b \in P\} \: \cup \: \{ (b, a_1]\colon b \in P\}
  \end{equation*}
  where $a_0$ is the smallest element in $P$ if existent and $a_1$ is the largest element in $P$ if existent.
\end{enumerate}
\end{ex}

\begin{defin}
  Let $X$ and $Y$ be topological spaces. A map $f\colon X \to Y$ is called \textbf{continuous} if for all $A \in \T_Y$ it follows that $f^{-1}(A) \in \T_X$.  
\end{defin}

\begin{defin}
  Let $X$ be a topological space and let $A \subseteq X$. A cover of $A$ is a collection of subsets $\left\{A_i \in \PowS(X)\colon i \in I \right\}$ with index set $I$ which has the following property
  \begin{equation}\label{eq:cover}
    \bigcup\limits_{i\in I}A_i = A.
  \end{equation}
  If the elements of the cover are open subsets of $X$ the cover is called an \textbf{open} cover.
  A subcollection of a cover of $A$ is called a \textbf{subcover} if it still fulfills Equation \ref{eq:cover}.
\end{defin}

\begin{defin}
  Let $X$ be a topological space and $A \subseteq X$. $A$ is called \textbf{compact} if every open cover of $A$ has a finite subcover. 
\end{defin}

\subsubsection{Filters}
In a metric space \((X, d_X)\), a sequence \((x_i)_{i \in \mathbb{N}}\) is said to converge to a limit point \(x \in X\) if, for every \(\varepsilon \in \R_{>0}\), there exists an \(N \in \mathbb{N}\) such that for all \(n \geq N\), the distance satisfies \(d_X(x, x_n) < \varepsilon\). This definition of convergence, however, relies on specific topological properties of metric spaces and does not extend naturally to general topological spaces. 

To address this limitation, topology introduces a more general concept of sequences known as \textbf{nets}. Nets extend the idea of sequences by allowing arbitrary index sets, making it possible to work with uncountable sequences. Despite this added flexibility, nets still share some of the same challenges as sequences—for example, selecting a convergent subsequence in a compact set.

Fortunately, an even more general notion of convergence exists in topology, which overcomes the limitations of both sequences and nets while still accommodating uncountable sequences. In the following this broader framework of convergence will be built up.

\begin{defin}\label{def:convtop}
  Let $X$ be a topological space, let $x \in X$ and let $(x_i)_{i \in \N}$ be a sequence in $X$. Then the sequence of the $x_n$ converges against $x$ ($x_n \to x$) if
  \begin{equation*}
    \forall U \in \mathcal{N}_x\exists N\in\N\forall n \geq N\colon x_n \in U.
  \end{equation*}

  Let $A \subseteq X$ and $x \in X$. Then $x$ is called an \textbf{accumulation point} of $A$ if for every $U \in \mathcal{N}_x$ we can find $y \in U \cap A$ with $y \neq x$.
\end{defin}
This is a more general definition of convergence of sequences working in abitrary topological spaces. The abstract definition of topological spaces makes this new notion of limits not as intuitive as in the special metric case. In some topological spaces there exist sequences that do not have a unique limit.

\begin{ex}
  Consider the topological space of the real numbers $\R$ but with the chaotic topology (see \ref{ex:top} \ref{ex:top-chaotic}). In this topological space, every sequence of numbers converges against every point. To this end consider a sequence $(x_i)_{i\in\N}$ in $\R$ and a real number $x$. We only have to check one open neighborhood of $x$ namely the set $\R$. And this trivially fulfills the condition of Definition \ref{def:convtop} because every sequence member is contained in it. So we have shown that an arbitrary sequence in this space converges against an arbitrary point.  
\end{ex}

This shows that additional assumptions have to be imposed onto a topological space such that the definition of convergence against a limit point makes sense.

\begin{defin}
  A topological space $X$ is called \textbf{Hausdorff} if for any two points $x, y \in X$ there exists an open neighborhood $U$ of $x$ and an open neighborhood $V$ of $y$ which are disjoint.   
\end{defin}

This property is enough such that every convergent sequence has a unique limit.

\begin{thm}
  Let $X$ be a topological space. If $X$ is Hausdorff then every convergent sequence in $X$ has a unique limit.
\end{thm}

\begin{proof}
  Let $(a_n)_{n \in \N}$ be a sequence in $X$ and suppose it converges against $x$ and $y$ in $X$ with $x \neq y$. 
  Because of the assumption that $X$ is Hausdorff there are open neighborhoods of the limit points $U$ and $V$ which are disjoint. 
  Now we know that there is are natural numbers $n_0$ and $n_1$ with the property
  \begin{equation*}
    \forall n \geq n_0\colon a_n \in U \: \land \: \forall n \geq n_1\colon a_n \in V
  \end{equation*}
  This means for $n := \max(n_0, n_1)$ that $a_n \in U \cap V = \emptyset \:\: \lightning$.
\end{proof}

The next example will show that this definition is not general enough for every topological space.

\begin{ex}[{\cite[5.3 Beispiel]{BvQMT}}]\label{ex:ordertop-uncount}
  Let $(\Omega, \geq)$ be an uncountable set which is well-ordered, has a biggest element $\omega_1$ and for all $\alpha \in \Omega$ with $\alpha < \omega_1$ the set $\{\beta \in \Omega\colon \beta \leq a\}$ is countable\footnote{This construction is possible because of the well-ordering pricinple. See \cite[p. 53]{BvQMT}}.
  Now define the topological space $\Omega$ with the order topology like in \ref{ex:top} \ref{ex:top-order} and $\Omega_0 := \Omega\setminus \{\omega_1\}$.
  It holds that $\omega_1$ is an accumulation point of $\Omega_0$ but there is no sequence in $\Omega_0$ that converges to $\omega_1$.
\end{ex}

\begin{proof}
  Suppose there is a sequence $(\alpha_n)_{n\in\N}$ in $\Omega_0$ such $\alpha_n \to \omega_1$. This means that $\sup_{n\in\N} a_n = \omega_1$. Now define
  \begin{equation*}
    A_n = \{\beta\colon \beta \leq a_n\}
  \end{equation*}
  for all $n \in \N$. Since the sets $A_n$ are all countable by definition of $\Omega$ the set
  \begin{equation*}
    B := \bigcup\limits_{n\in\N} A_n = \{\beta \in \Omega\colon \exists m\in\N \colon \beta \leq a_m \}
  \end{equation*}
  is also countable. This means the smallest element of $\Omega\setminus B$ is well-defined. Call it $\gamma$. Thus
  \begin{equation*}
    \beta \in B \iff \beta \leq \gamma. 
  \end{equation*}
  But by definition of $\Omega$ and the fact that $\gamma \in \Omega_0$ it follows that $\gamma < \omega_1$. Now we get
  \begin{equation*}
    \sup\limits_{n\in\N} a_n \leq \gamma < \omega_1. \:\: \lightning \qedhere
  \end{equation*}
\end{proof}

This example illustrates the fact that not in every topological space, the standard definition of a sequence and convergence of a sequence is enough. There can be accumulation points that cannot be reached by any sequence. The problem is that the element $\omega_1$ has an uncountable neighborhood basis (see \ref{defin:nbs}) and sequences have only countable many elements such that the definition of convergence cannot be fulfilled by a sequence because of cardinality reasons. This problem cannot arise in metric spaces because the countability of all neighborhood basis is a condition for a topological space to be metrizable (see \cite[p. 130f]{MunTop}).

In the following a more general definition of convergence in a topological space is developed.
\begin{defin}
  A \textbf{filter} $\F$ on a non-empty set $X$ is a collection of subsets of $X$ such that:
  \begin{enumerate}
    \item $X \in \F$, $\emptyset \notin \F$,
    \item $\forall A, B \in \F\colon A\cap B \in \F$,
    \item $\forall A \subseteq B \subseteq X: A \in \F \Rightarrow B \in \F$.
  \end{enumerate}
  Denote by ${\rm Flt}(X)$ the set of all filters on $X$.
  Let $\F'$ be another filter on $X$. If $\F \subseteq \F'$ we call $\F'$ finer than $\F$ or $\F$ coarser then $\F'$. A family of subsets $\mathcal{F}_0 \subseteq \mathcal{F}$ of $X$ is called a \textbf{filter basis} of $\mathcal{F}$ if
  \begin{equation*}
    \forall F \in \mathcal{F} \exists F_0 \in \mathcal{F}_0\colon F_0 \subseteq F. \qedhere
  \end{equation*}
\end{defin}

\begin{defin}\label{defin:nbs}
  Let $X$ be a topological space. The \textbf{neighborhood basis} at $x \in X$ is a filter basis of the neighborhood filter $\mathcal{N}_X(x)$.
\end{defin}
  
\begin{lemma}\label{lem:filbas}
  Let $\mathcal{B}$ be a collection of non-empty subsets of a non-empty set $X$. $\mathcal{B}$ is a basis of a filter if and only if
  \begin{equation}\label{eq:filterbasis}
    \forall B_1, B_2 \in \mathcal{B}\exists B_3 \in \mathcal{B}: B_3 \subseteq B_1 \cap B_2.
  \end{equation}
\end{lemma}

\begin{proof}
  Assume $\mathcal{B}$ is the basis of a filter $\mathcal{F}$ on $X$. It holds especially that $\mathcal{B} \subseteq \mathcal{F}$. Let $B_1, B_2 \in \mathcal{B}$. Since these are also filter elements it follows that $B_1 \cap B_2 \in \mathcal{B}$ which proves the claim.

  Now assume that $\mathcal{B}$ fulfills equations \ref{eq:filterbasis} and let $B_1, B_2 \in \mathcal{B}$. Define the filter \[\mathcal{F} := \{ F \in \PowS(X)\setminus\{ \emptyset \}\colon \exists B\in\mathcal{B}\colon B \subseteq F \}.\] It is clear that $\emptyset \notin \mathcal{F} \ni X$. Furthermore this filter is closed under the operation of taking supersets by definition. Now suppose that $F_1, F_2 \in \mathcal{F}$. There exist $B_1, B_2 \in \mathcal{B}$ with $B_1 \subseteq F_1$ and $B_2 \subseteq F_2$. Now by Equation \ref{eq:filterbasis} it follows that there is $B_3 \in \mathcal{B}$ with $B_3 \subseteq B_1 \cap B_2$ and thus
  \begin{equation*}
    \emptyset \neq B_3 \subseteq B_1 \cap B_2 \subseteq F_1 \cap F_2.
  \end{equation*}
  Hence $F_1 \cap F_2 \in \mathcal{F}$.
\end{proof}

From this point the notation $\mathcal{F}(\mathcal{B}) := \{F \in \PowS(X)\setminus\{\emptyset\}\colon \exists B\in\mathcal{B}\colon B\subseteq F\}$ represents the filter on $X$ generated by the filter basis $\mathcal{B}$.

\begin{col}\label{cor:filbas}
  Let $X$ be a non-empty set, $A \subseteq X$ and let $\mathcal{F}$ be a filter on $X$. If $A \cap F \neq \emptyset$ for all $F\in\mathcal{F}$ then the set $\mathcal{A} := \{A \cap F\colon F\in \mathcal{F}\}$ forms a filter basis of a filter which is finer than $\mathcal{F}$.
\end{col}

\begin{proof}
  Let $F_1, F_2 \in \mathcal{F}$ and define $A_1 := F_1 \cap A$ and $A_2 := F_2 \cap A$. Now consider
  \begin{equation*}
    A_1 \cap A_2 = (F_1 \cap A) \cap (F_2 \cap A) = F_1 \cap F_2 \cap A = F_3 \cap A \in \mathcal{A}
  \end{equation*}
  where $F_3 = F_1 \cap F_2 \in \mathcal{F}$. Thus $\mathcal{A}$ is a filter basis by Lemma \ref{lem:filbas}.

  Now consider the filter $\mathcal{F}(\mathcal{A})$. Since $\mathcal{A}$ is a filter basis of this filter all elements of the form $F \cap A$ for $F\in\mathcal{F}$ are contained in $\mathcal{F}(\mathcal{A})$. For each $F\in\mathcal{F}$ there exists $F\cap A\in \mathcal{F}(\mathcal{A})$ and since $F\cap A \subseteq F$ it follows that $F\in \mathcal{F}(\mathcal{A})$. 
\end{proof}

\begin{defin} 
  An \textbf{ultrafilter} $\F$ on $X$ is a filter on $X$ with the property that if there is another filter on X called $\F'$ such that $\F \subseteq \F'$ it follows that $\F = \F'$. Denote by ${\rm UFlt}(X)$ the set of all ultrafilters on $X$.
  If $\bigcap \F = \emptyset$ the ultrafilter is called \textbf{free}.
\end{defin}

\begin{lemma}\label{lem:ufltlemma}
  Let $X$ be a non-empty set and let $\mathcal{F} \in {\rm UFlt(X)}$ then
  \begin{equation*}
    \forall A \subseteq X\colon A \in \mathcal{F} \lor A^c \in \mathcal{F}.
  \end{equation*}
\end{lemma}

\begin{proof}
  Since $A \cap A^c = \emptyset$ there are no two sets $F_1, F_2 \in \mathcal{F}$ with $F_1 \subseteq A$ and $F_2 \subseteq A^c$. This means that for all $F \in \mathcal{F}$ either $F \cap A \neq \emptyset$ or $F \cap A^c \neq \emptyset$. Assume w.l.o.g. that $A \cap F \neq \emptyset$ for all $F \in \mathcal{F}$. From this it follows that $\{F \cap A\colon F\in \mathcal{F}\}$ is a filter basis of a filter $\mathcal{G}$ which is finer than $\mathcal{F}$ by Corollary \ref{cor:filbas}. Since $\mathcal{F}$ is an ultrafilter it follows that $\mathcal{F} = \mathcal{G}$ and thus $A \in \mathcal{F}$. \cite[5.12 Satz]{BvQMT}
\end{proof}

\begin{defin}
  Let $X$ be a topological space. A filter $\F$ converges to an element $x$ in $X$ ($\F \to x$) if $\mathcal{N}_x \subseteq \F$. 
  
  Let $Y$ be another topological space and $f: X \to Y$ a continuous function. 
  We call $f(\F)$ the \textbf{image filter} of $\F$ under $f$ which is the filter with filter basis $\{f(F)\colon F \in \F\}$. Another notation for the convergence of a filter is $\lim\limits_{F\to\F}f(F)$ (instead of $f(\F) \to y$).
\end{defin}

Indeed the new definition of convergence extends the old one from \ref{def:convtop}. Let $X$ be a topological space and let $(x_n)_{n\in\N}$ be a sequence in $X$ which converges to $x \in X$. Then there exists a filter on $X$ which converges to $x$ namely $\mathcal{N}_X(x)$. The inverse is not true because the new definition of convergence now allows the convergence to the point $\omega_1$ in Example \ref{ex:ordertop-uncount}. And it is also true that if $A \subseteq X$ and $a \in \bar{A}$ then there exists a filter that converges to $a$. This is not true for sequences by Example \ref{ex:ordertop-uncount}.

\begin{thm}\label{thm:fcl}
  Let $X$ be a topological space and let $A \subseteq X$. Then
  \begin{equation*}
    x \in \bar{A} \: \iff \: \exists \F \in {\rm Flt}(X): A \in \F \: \land \: \F \to x.
  \end{equation*}
\end{thm}

\begin{proof}
  See \cite[5.17 Satz]{BvQMT}.
\end{proof}

\begin{defin}
  The \textbf{Cofinite-Filter} $\F_{CF}$ on an infinite set $X$ is defined as 
  \begin{equation*}
    \F_{CF} := \{ F \in \P(X)\colon \left| F^c \right| < \infty\}.\qedhere
  \end{equation*}
\end{defin}

\begin{lemma}\label{lem:coffil}
  Let $X$ be an infinite set, then the Cofinite-Filter on $X$ is a free filter.
\end{lemma}

\begin{proof}
  It is trivial to see that $X \in \F_{CF}\:$ and $\:\emptyset \notin \F_{CF}$. So let $A, B \in \F_{CF}$ then
  \begin{align*}
    (A \cap B)^c = A^c \cup B^c \Rightarrow \left| (A \cap B)^c \right| = \left | A^c \cup B^c \right| \leq \left| A^c \right| + \left| B^c \right| < \infty
  \end{align*}
  since $A^c$ and $B^c$ are finite.
  Now assume that $A \subseteq B$. It follows that
  \begin{align*}
    A \subseteq B \: \Rightarrow \: B^c \subseteq A^c \: \Rightarrow \left|B^c\right| \leq \left|A^c\right| < \infty.
  \end{align*}
  This proves that $\F_{CF}$ is a filter on $X$.
  Now assume that $\bigcap \F_{CF}$ is non-emtpy and let $x \in \bigcap \F_{CF}$. Now let $F \in \F_{CF}$. By assumption it follows that $x \in F$ and
  \begin{align*}
    \left|(F\setminus \{x\})^c\right| = \left| F^c \right| + \left| \{x \} \right| = \left| F^c \right| + 1 < \infty
  \end{align*}
  and hence $F\setminus \{x\} \in \F_{CF}$. $\lightning$
\end{proof}

\begin{thm}\label{thm:ulfil}
  Let $X$ be non-empty set. Every filter on X is contained in an ultrafiler on $X$.
\end{thm}

\begin{proof}
  Let $\F$ be a filter on $X$ and let $\Phi$ be the set of all filters on $X$ that are finer than $\F$. This set forms a partial order together with the $\subseteq$ relation.
  If $\Phi_1$ is a totally ordered subset of $\Phi$ then define $\F' := \bigcup\limits_{\mathcal{G} \in \Phi} \mathcal{G}$.
  \textbf{Claim: } $\F'$ is a filter on $X$. The first property is clear because every filter contains $X$ and does not contain $\emptyset$.
  Now let $A, B \in \F'$. This means that there are filters $\mathcal{G}_1$ and $\mathcal{G}_2$ in $\F'$ with $A \in \mathcal{G}_1$ and $B \in \mathcal{G}_2$.
  Since $\mathcal{G}_1 \in \Phi_1$ and $\mathcal{G}_2 \in \Phi_1$ it follows that $\mathcal{G}_1 \subseteq \mathcal{G}_2$ or $\mathcal{G}_2 \subseteq \mathcal{G}_1$. Suppose w.l.o.g. that $\mathcal{G}_1 \subseteq \mathcal{G}_2$.
  It follows that
  \begin{equation*}
    A \in \mathcal{G}_2 \Rightarrow A \cup B \in \mathcal{G}_2 \Rightarrow A \cup B \in \F'.
  \end{equation*}
  Now suppose that $A \in \F'$ and $A \subseteq B \subseteq X$. Since there is an $\mathcal{G} \in \Phi_1$ with $A \in \mathcal{G}$ we can conclude that
  \begin{equation*}
    B \in \mathcal{G} \Rightarrow B \in \F'.
  \end{equation*}
  This proves the claim. $\F'$ is an upper bound of $\Phi_1$ and thus $\Phi_1$ is inductively sorted. The existence of an ultrafilter on $X$ follows from Zorn's Lemma. \cite[5.12 Satz]{BvQMT}
\end{proof}

\begin{col}\label{col:exfreeuf}
  Let $X$ be a infinite set. There exists a free ultrafilter on $X$.
\end{col}

\begin{proof}
  Follows directly from \ref{lem:coffil} and \ref{thm:ulfil}.
\end{proof}

\begin{lemma}
  Every ultrafilter on a non-empty finite set $X$ converges to a point of $X$.
\end{lemma}

\begin{proof}
  In the case of a discrete set, the ultrafilter converges against an element if the singleton set containing this element is contained inside of the ultrafilter.
  Assume the cardinality of $X$ is 1. Then the claim follows directly by the definition of the ultrafilter.
Now assume that the Lemma is true for a set with cardinality $n \in \N$ and let $X$ be a set with cardinality $n+1$ and additionally let $\mathcal{F}$ be an ultrafilter on $X$. Write $X = \{1, \ldots, n\} \cup \{n+1\}$. By Theorem \ref{lem:ufltlemma} either $\{n+1\} \in \mathcal{F}$ which ends the proof by the definition of convergence of an ultrafilter or $\{1,\ldots,n\} \in \mathcal{F}$ which ends the proof by the inductive assumption.
\end{proof}

\subsection{Topological Algebra}

\begin{defin}
  Let $G$ be a group with group operation $\cdot_G$ and $\T$ a topology on the underlying set of $G$. Additionally define ${\rm mul}\colon G \times G \to G, (g, h) \mapsto g \cdot_G h$ as the multiplication map of $G$ and ${\rm inv}\colon G \to G, g \mapsto g^{-1}$ as the inverse map of $G$. Then $G$ is called a \textbf{topological group} if both ${\rm mul}$ and ${\rm inv}$ are continous maps with respect to $\T$.
  Let $X$ be a set. An action of $G$ on $X$ is a map $\lambda\colon G\times X \to X, (g, x) \mapsto \lambda(g, x) =: g \anddot x$ that fulfills the following properties
  \begin{enumerate}
    \item $\forall g, h \in G: h \anddot (g \anddot x) = (h \cdot_G g) \anddot x$,
    \item $e_G \in G$: $\lambda(e, x) = x$ for all $x \in X$.
  \end{enumerate}
  If $X$ is a topological space, then the action of $G$ on $X$ is continuous if $\lambda$ is a continuous map.
\end{defin}

\begin{defin}
  A topological group $G$ is called \textbf{extremely amenable} if every continous action of G on a compact space admits a fixed point.
\end{defin}

\begin{defin}
  Let $X$ and $Y$ be sets. A step function $f\colon X \to Y$ with finite range induces a finite partion on X called $\P_f$ in the following way
  \begin{equation*}
    \P_f := \{f^{-1}(\{y\})\colon \forall y \in Y\}.
  \end{equation*}
  This is a well-defined finite partion because of the assumption that $f$ has finite range.
\end{defin}

\begin{defin}\label{defin:sf}
  Let $X$ be a set and let $\mu\colon \mathcal{B} \to [0, \infty)$ be a submeasure on $X$ on the boolean algebra $\mathcal{B} \subseteq \PowS(X)$. Then the set
  \begin{equation*}
    S(\mu, G) := \left\{ f: X \to G\colon f \:\:\mu-\text{measureable} \: \land \: \left| f(X) \right| < \infty \right\}
  \end{equation*}
  for a topological group $G$ is the set of all measurable step functions with finite range on $X$ and values in $G$.
\end{defin}

\begin{thm}\label{thm:stop}
  The set $S(\mu, G)$ for a submeasure $\mu$ on a set $X$ and an abelian topological group $G$ is a group with group operation
  \begin{equation*}
    \star\colon S(\mu, G) \times S(\mu, G) \to S(\mu, G), (f, g) \mapsto f \star g
  \end{equation*}
  where $(f\star g)(x) = f(x) \cdot_G g(x)$ for each $x \in X$. If $f$ is an element of $S(\mu, G)$ then define $f^{-1}$ as $f^{-1}(x) = (f(x))^{-1}$ for all $x \in X$. If we put the topology of convergence in submeasure on $S(\mu, G)$ the group
  becomes a topological group.

  A basis element of the topology of convergence in submeasure is given by
  \begin{equation*}
    V_\varepsilon(f) := \{ h \in S(\mu, G)\colon \mu(\{x \in X\colon h(x) \notin V\cdot_G f(x)\}) < \varepsilon \}
  \end{equation*}
  where $V$ is a neighborhood of the identity on $G$, $f \in S(\mu, G)$ and $\varepsilon \in \R_{>0}$.
\end{thm}

\begin{proof}
  Firstly let $f, g \in S(\mu, G)$. Since $f$ and $g$ induce partitions $\P_f, \P_g$ and the partition $\P = \{A \cap B\colon A \in \P_f, B \in \P_g\}$ is again a measurable partition it is clear that $f \star g \in S(\mu, G)$. The associativity, existence of a neutral element and existence of an inverse element follow from the fact that $\cdot_G$ is a group operation.
  It remains to show that ${\rm mul}\colon S(\mu, G)\times S(\mu, G) \to S(\mu, G), (f, g) \mapsto f\star g$ and ${\rm inv}\colon S(\mu, G) \to S(\mu, G), f \mapsto f^{-1}$ are continuous. To this end let $V$ be a neighborhood of the identity of $G$, $f \in S(\mu, G)$ and let $\varepsilon \in \R_{>0}$. Consider
  \begin{align*}
    {\rm inv}^{-1}(V_\varepsilon(f)) &= \{h \in S(\mu, G)\colon \mu(\{x \in X\colon (h(x))^{-1} \notin V\cdot f(x)\}) < \varepsilon\} \\
    &= \{h \in S(\mu, G)\colon \mu(\{x \in X\colon h(x) \notin V^{-1}\cdot (f(x))^{-1}\}) < \varepsilon\}
  \end{align*}
  where $V^{-1} := \{g^{-1}\colon g \in V\}$.
  This is a neighborhood of $f^{-1}$ since $V^{-1}$ is also a neighborhood of the identity of $G$. 
  Now consider $M := {\rm mul}^{-1}(V_\varepsilon(f))$ which is the set \[M = \{(h, g) \in S(\mu, G) \times S(\mu, G)\colon \mu(\{x \in X\colon (h \star g)(x) \notin V\cdot f(x)\}) < \varepsilon\}.\]
  This set is open since
  \begin{align*}
    \pi_1(M) &= \{h \in S(\mu, G)\colon \exists g\in S(\mu, G)\colon \mu(\{x \in X\colon (h \star g)(x) \notin V\cdot f(x)\}) < \varepsilon\} \\
    &= \{h \in S(\mu, G)\colon \exists g\in S(\mu, G)\colon \mu(\{x \in X\colon h(x) \cdot g(x) \notin V\cdot f(x)\}) < \varepsilon\} \\
    &= \{h \in S(\mu, G)\colon \exists g\in S(\mu, G)\colon \mu(\{x \in X\colon h(x) \notin V\cdot f(x) \cdot (g(x))^{-1}\}) < \varepsilon\} \\
    &= \{h \in S(\mu, G)\colon \exists \tilde{f}\in S(\mu, G)\colon \mu(\{x \in X\colon h(x) \notin V\cdot \tilde{f}(x)\}) < \varepsilon\} \\
    &= \bigcup\limits_{\tilde{f} \in S(\mu, G)}\{h \in S(\mu, G)\colon \mu(\{x \in X\colon h(x) \notin V\cdot \tilde{f}(x)\}) < \varepsilon\}
  \end{align*}
  where $\pi_1$ is the projection to the first component. By symmetry reasons this also works for the second component.
\end{proof}

\subsection{Algebraic Topology}
In the following section let $Y$ be a topological space and let $X$ be a metric space with metric $d_X$.

\begin{defin}
  A \textbf{path} $\gamma$ in $Y$ is a continuous map $[0,1] \to Y$ where $[0,1]$ is seen as a subspace of $\R$.
\end{defin}

\begin{defin}
  Let $\gamma_1, \gamma_2 \in C([0, 1], Y)$. A \textbf{free homotopy} $H\colon [0,1]^2 \to Y$ between $\gamma_1$ and $\gamma_2$ is a continuous map such that
  \begin{equation*}
    \forall s \in [0,1]\colon H(s, 0) = \gamma_1(s) \: \land \: H(s, 1) = \gamma_2(s)
  \end{equation*}

  Now let $\gamma_1, \gamma_2 \in C([0, 1], Y)$ such that $\gamma_1(0) = \gamma_2(0)$ and $\gamma_1(1) = \gamma_2(1)$ i.e. the two paths start and end at the same point.
  The homotopy $H \in C([0,1]^2, Y)$ is called \textbf{endpoint preserving} if it is a free homotopy between $\gamma_1$ and $\gamma_2$ and it has the following property:
  \begin{equation*}
    \forall t\in[0,1]\forall i\in \{1,2\}\colon H(0, t) = \gamma_i(0) \: \land \: H(1, t) = \gamma_i(1).
  \end{equation*}

  If a free homotopy $H$ between the two paths $\gamma_1$ and $\gamma_2$ exists they are called \textbf{freely homotopic} $\gamma_1 \overset{\cdot}{\sim}_H \gamma_2$.
  If the two paths $\gamma_1$ and $\gamma_2$ have the same start and end point and there exists an endpoint preserving homotopy $H$ between them then they are called 
  \textbf{homotopic} $\gamma_1 \sim_H \gamma_2$. If a path is (freely) homotopic to a constant path ($[0,1] \to Y, t \mapsto y \in Y$) then it is called \textbf{null-homotopic}.

  Lastly define the homotopy relation $\sim$ on $C([0,1], Y)$ where 
  \begin{equation*}
    \forall\gamma_1, \gamma_2 \in C([0,1], Y)\colon \gamma_1 \sim \gamma_2 \iff \exists H \text{ homotopy between } \gamma_1 \text{ and } \gamma_2.
  \end{equation*}
\end{defin}

In this thesis all homotopies are endpoint preserving if it is not explicitly stated that they should be free homotopies.

\begin{lemma} \label{lem:homotopy-equivalence}
  The homotopy relation $\sim$ of paths is an equivalence relation on the set $C([0,1],Y)$.
\end{lemma}

\begin{proof}
  Let $\gamma_1, \gamma_2, \gamma_3 \in C([0,1],Y)$ with $\gamma_1(0) = \gamma_2(0) = \gamma_3(0)$ and $\gamma_1(1) = \gamma_2(1) = \gamma_3(1)$.
  
  \textit{Reflexivity:}
  Consider the homotopy $H\colon [0,1]^2 \to Y, \: (s, t) \mapsto \gamma_1(s)$. 
  
  It holds that for all $s\in [0,1]\colon H(s, 0) = \gamma_1(s),\: H(s, 1) = \gamma_1(s)$. And thus $\gamma_1 \sim_H \gamma_1 \Rightarrow \gamma_1 \sim \gamma_1$.

  \textit{Symmetry:} 
  Assume that $\gamma_1 \sim \gamma_2$. This means there exists a homotopy $H$ such that $\gamma_1 \sim_H \gamma_2$. 
  Now define the homotopy $F\colon [0,1]^2 \to Y, \: (s, t) \mapsto H(s, 1-t)$. 
  
  From this definition it follows that for all $s \in [0,1]:$
  \begin{align*}
    F(s, 0) &= H(s, 1 - 0) = \gamma_2(s) \\
    F(s, 1) &= H(s, 1 - 1) = \gamma_1(s)
  \end{align*}
  and hence $\gamma_2 \sim_F \gamma_1 \Rightarrow \gamma_2 \sim \gamma_1$.

  \textit{Transitivity:}
  Assume that $\gamma_1 \sim \gamma_2$ and $\gamma_2 \sim \gamma_3$. Let $H$ be a homotopy between $\gamma_1$ and $\gamma_2$ and let $F$ be a homotopy between $\gamma_2$ and $\gamma_3$. 
  Define the homotopy
  \begin{equation*} 
    G: [0,1]^2 \to Y, \: (s,t) \mapsto \begin{cases}
      H(s, 2t), &t \in [0, {1 \over 2}], \\
      F(s, 2t - 1), &t \in [{1 \over 2}, 1].
    \end{cases}
  \end{equation*}

  Then the following equations hold
  \begin{align*}
    G(s,0) &= H(s,0) = \gamma_1(s), \\
    G\left(s, \tfrac{1}{2}\right) &= H(s, 1) = \gamma_2(s) = F(s, 0), \\
    G(s, 1) &= F(s, 1) = \gamma_3(s),
  \end{align*}
  and thus $\gamma_1 \sim_G \gamma_3 \Rightarrow \gamma_1 \sim \gamma_3$.
\end{proof}

\begin{defin}
  Let $X$ and $Y$ be topological spaces. The \textbf{homotopy class} of $f \in C(X, Y)$ is $[f]_{\sim} = \{g \in C(X, Y)\colon f \simeq g\}$. A continuous function $f \in C(X, Y)$ is called a \textbf{homotopy equivalence} if there exists a $g \in C(Y, X)$ such that
  \begin{equation*}
    g \circ f \simeq \id_X \: \land \: f \circ g \simeq \id_Y.
  \end{equation*}
  The two spaces $X$ and $Y$ are \textbf{homotopy equivalent} ($X \simeq Y$) if there exists a homotopy equivalence $f \in C(X, Y)$ between them.

  A space is called \textbf{contractible} if it is homotopy equivalent to a one point space.
\end{defin}

\begin{lemma}
  Let $X$ and $Y$ be topological spaces, then
  \begin{equation*}
    X \cong Y \: \Rightarrow \: X \simeq Y.
  \end{equation*}
\end{lemma}

\begin{proof}
  Let $\psi\colon X \to Y$ be a homeomorphism. Then it follows that $\psi \circ \psi^{-1} \simeq \id_Y$ and $\psi^{–1} \circ \psi \simeq \id_X$.  
\end{proof}
