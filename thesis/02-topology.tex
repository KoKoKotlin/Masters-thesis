\section{Topology, Topological Algebra and Algebraic Topology}
\subsection{Topology}

\subsubsection{Topological Spaces}
\begin{defin}
  Let $X$ be a set. A \textbf{topology} $\T$ on $X$ is a collection of subsets of $X$ obeying the following axioms
  \begin{enumerate}
    \item $X \in \T \ni \emptyset$,
    \item $\bigcup\limits_{\alpha \in I}A_{\alpha} \in \T$, where $I$ is an arbitrary index set and $A_{\alpha} \in \T$ for all $\alpha \in I$,
    \item $\bigcap\limits_{i=0}^n A_i \in \T$ for $n \in \N$ and $A_i \in \T$.
  \end{enumerate}
  
  The tuple $(X, \T)$ where $\T$ is a topology on $X$, is called a \textbf{topological space}.
\end{defin}

In the following we will say that $X$ is a topological space without mentioning the explicit topology every time. 

\begin{defin}
  Let $X$ be a topological space. A family of open sets $\mathcal{B} \subseteq \mathcal{T}_X$ is called a \textbf{basis} of the topology on $X$ if
  \begin{equation*}
    \forall U \in \mathcal{T}_X\forall x \in U\exists B \in\mathcal{B}\colon x \in B \subseteq U.
  \end{equation*}
  A family of open sets $\mathcal{S} \subseteq \mathcal{T}_X$ is called a \textbf{subbasis} of the topology on $X$ if
  \begin{equation*}
    \left\{\bigcap \mathcal{S}'\colon \mathcal{S}'\in \PowSF(\mathcal{S})\right\}
  \end{equation*}
  is a basis of the topology on $X$.
\end{defin}
The intuitive idea of the definition above is that every open set in the topological space $X$ can be represented by an arbitrary union of basis elements. Additionally each basis element is only a finite intersection of subbasis elements.

In many cases it is not possible to describe the topology on $X$ directly and one can only give a basis for the topology. Then one can talk about $\mathcal{T}(\mathcal{B})$ which is the topology generated by the set $\mathcal{B}$. This topology consists of all arbitrary unions of elements in $\mathcal{B}$. The next lemma shows that not every subset of $\PowS(X)$ is a basis for a well-defined topology topology on $X$.
\begin{lemma}
  Let $X$ be a set and let $\mathcal{B} \subseteq \PowS(X)$, then the following are equivalent:
  \begin{itemize}
  \item $\mathcal{B}$ is the basis of a topology on $X$,
    \item $\bigcup \mathcal{B} = X$ and $\forall B_1, B_2 \in \mathcal{B}\exists B_3 \in \mathcal{B}\colon \forall x \in B_1\cap B_2\colon x \in B_3 \subseteq B_1\cap B_2$.
  \end{itemize}
\end{lemma}

\begin{proof}
  See \cite[p. 78f.]{MunTop}.
\end{proof}

\begin{ex}\label{ex:top}
  Here are some examples of topological spaces which will be needed later.
\begin{enumerate}[label=\roman*.)]
  \item \label{ex:top-chaotic} Let $X$ be a nonempty set. Then the set $\T_X := \{X, \emptyset\}$ is a topology on $X$ called the \textbf{chaotic topology}.
  \item \label{ex:top-order} Let $(P, <)$ be a totally ordered set\footnote{see \cite[p. 24]{MunTop} for definition of total order (there simple order)}. Define for $a, b \in P$ with $a < b$ the \textbf{open interval} from a to b as $(a,b) := \{ x \in X\colon a < x \: \land \: x < b \}$ and similarly define the half open intervals $[a,b) := \{a\} \cup (a,b)$ and $(a,b] := (a,b) \cup \{ b \}$. 
  Then the \textbf{order topology} on $P$ is generated by the basis
  \begin{equation*}
    \mathcal{B} := \{ (a,b) \colon a, b \in P \} \: \cup \: \{[a_0, b)\colon b \in P\} \: \cup \: \{ (b, a_1]\colon b \in P\}
  \end{equation*}
  where $a_0$ is the smallest element in $P$ if existent and $a_1$ is the largest element in $P$ if existent.
\end{enumerate}
\end{ex}

\begin{defin}
  Let $X$ and $Y$ be topological spaces. A map $f\colon X \to Y$ is called \textbf{continuous} if for all $A \in \T_Y$ it follows that $f^{-1}(A) \in \T_X$.  
\end{defin}

\begin{defin}
  Let $X$ be a topological space and let $A \subseteq X$. A collection of subsets $(A_i)_{i\in I}$ of $A$ where $A_i \in \PowS(X)$ and $I$ is an arbitrary index set is called a \textbf{cover} of $A$ if it fulfills the following property
  \begin{equation}\label{eq:cover}
    \bigcup\limits_{i\in I}A_i = A.
  \end{equation}
  If the elements of the cover are open subsets of $X$ the cover is called an \textbf{open} cover.
  A subcollection of a cover of $A$ is called a \textbf{subcover} if it still fulfills Equation \ref{eq:cover}.
\end{defin}

\begin{defin}
  Let $X$ be a topological space and $A \subseteq X$. $A$ is called \textbf{compact} if every open cover of $A$ has a finite subcover. 

  Furthermore $X$ is called \textbf{locally-compact} if there exists a neighborhood basis of compact sets at each point $x \in X$. 
\end{defin}

\subsubsection{Filters}
In a metric space \((X, d_X)\), a sequence \((x_i)_{i \in \mathbb{N}}\) is said to converge to a limit point \(x \in X\) if, for every \(\varepsilon \in \R_{>0}\), there exists an \(N \in \mathbb{N}\) such that for all \(n \geq N\), the distance satisfies \(d_X(x, x_n) < \varepsilon\). This definition of convergence, however, relies on specific topological properties of metric spaces and does not extend naturally to general topological spaces. 

To address this limitation, topology introduces a more general concept of sequences known as \textbf{nets}. Nets extend the idea of sequences by allowing arbitrary index sets, making it possible to work with uncountable sequences. Despite this added flexibility, nets still share some of the same challenges as sequences—for example, selecting a convergent subsequence in a compact set.

Fortunately, an even more general notion of convergence exists in topology, which overcomes the limitations of both sequences and nets while still accommodating uncountable sequences. In the following this broader framework of convergence will be built up.

\begin{defin}\label{def:convtop}
  Let $X$ be a topological space, let $x \in X$ and let $(x_i)_{i \in \N}$ be a sequence in $X$. Then the sequence of the $x_n$ converges to $x$ ($x_n \to x$) if
  \begin{equation*}
    \forall U \in \mathcal{N}_x\exists N\in\N\forall n \geq N\colon x_n \in U.
  \end{equation*}

  Let $A \subseteq X$ and $x \in X$. Then $x$ is called an \textbf{accumulation point} of $A$ if for every $U \in \mathcal{N}_x$ we can find $y \in U \cap A$ with $y \neq x$.
\end{defin}
This is a more general definition of convergence of sequences working in arbitrary topological spaces. The abstract definition of topological spaces makes this new notion of limits not as intuitive as in the special metric case. In some topological spaces there exist sequences that do not have a unique limit.

\begin{ex}
  Consider the topological space of the real numbers $\R$ but with the chaotic topology (see \ref{ex:top} \ref{ex:top-chaotic}). In this topological space, every sequence of numbers converges against every point. To this end consider a sequence $(x_i)_{i\in\N}$ in $\R$ and a real number $x$. We only have to check one open neighborhood of $x$ namely the set $\R$. And this trivially fulfills the condition of Definition \ref{def:convtop} because every sequence member is contained in it. So we have shown that an arbitrary sequence in this space converges against an arbitrary point.  
\end{ex}

This shows that additional assumptions have to be imposed onto a topological space such that the definition of convergence against a limit point makes sense.

\begin{defin}
  A topological space $X$ is called \textbf{Hausdorff} if for any two points $x, y \in X$ there exists an open neighborhood $U$ of $x$ and an open neighborhood $V$ of $y$ which are disjoint.   
\end{defin}

This property is enough such that every convergent sequence has a unique limit.

\begin{thm}
  Let $X$ be a topological space. If $X$ is Hausdorff then every convergent sequence in $X$ has a unique limit.
\end{thm}

\begin{proof}
  Let $(a_n)_{n \in \N}$ be a sequence in $X$ and suppose it converges against $x$ and $y$ in $X$ with $x \neq y$. 
  Because of the assumption that $X$ is Hausdorff there are open neighborhoods of the limit points $U$ and $V$ which are disjoint. 
  Now we know that there are natural numbers $n_0$ and $n_1$ with the property
  \begin{equation*}
    \forall n \geq n_0\colon a_n \in U \: \land \: \forall n \geq n_1\colon a_n \in V.
  \end{equation*}
  This means for $n := \max(n_0, n_1)$ that $a_n \in U \cap V = \emptyset \:\: \lightning$.
\end{proof}

The next example will show that this definition is not general enough for every topological space.

\begin{ex}[{\cite[5.3 Beispiel]{BvQMT}}]\label{ex:ordertop-uncount}
  Let $(\Omega, \geq)$ be an uncountable set which is well-ordered, has a biggest element $\omega_1$ and for all $\alpha \in \Omega$ with $\alpha < \omega_1$ the set $\{\beta \in \Omega\colon \beta \leq a\}$ is countable\footnote{This construction is possible because of the well-ordering principle. See \cite[p. 53]{BvQMT}}.
  Now define the topological space $\Omega$ with the order topology like in \ref{ex:top} \ref{ex:top-order} and $\Omega_0 := \Omega\setminus \{\omega_1\}$.
  It holds that $\omega_1$ is an accumulation point of $\Omega_0$ but there is no sequence in $\Omega_0$ that converges to $\omega_1$.
\end{ex}

\begin{proof}
  Suppose there is a sequence $(\alpha_n)_{n\in\N}$ in $\Omega_0$ such $\alpha_n \to \omega_1$. This means that $\sup_{n\in\N} a_n = \omega_1$. Now define
  \begin{equation*}
    A_n = \{\beta\colon \beta \leq a_n\}
  \end{equation*}
  for all $n \in \N$. Since the sets $A_n$ are all countable by definition of $\Omega$ the set
  \begin{equation*}
    B := \bigcup\limits_{n\in\N} A_n = \{\beta \in \Omega\colon \exists m\in\N \colon \beta \leq a_m \}
  \end{equation*}
  is also countable. This means the smallest element of $\Omega\setminus B$ is well-defined. Call it $\gamma$. Thus
  \begin{equation*}
    \beta \in B \iff \beta \leq \gamma. 
  \end{equation*}
  But by definition of $\Omega$ and the fact that $\gamma \in \Omega_0$ it follows that $\gamma < \omega_1$. Now we get
  \begin{equation*}
    \sup\limits_{n\in\N} a_n \leq \gamma < \omega_1. \:\: \lightning \qedhere
  \end{equation*}
\end{proof}

This example illustrates the fact that not in every topological space, the standard definition of a sequence and convergence of a sequence is enough. There can be accumulation points that cannot be reached by any sequence. The problem is that the element $\omega_1$ has an uncountable neighborhood basis (see \ref{defin:nbs}) and sequences have only countable many elements such that the definition of convergence cannot be fulfilled by a sequence because of cardinality reasons. This problem cannot arise in metric spaces because the countability of all neighborhood basis is a condition for a topological space to be metrizable (see \cite[p. 130f]{MunTop}).

In the following a more general definition of convergence in a topological space is developed.
\begin{defin}
  A \textbf{filter} $\F$ on a non-empty set $X$ is a collection of subsets of $X$ such that:
  \begin{enumerate}
    \item $X \in \F$, $\emptyset \notin \F$,
    \item $\forall A, B \in \F\colon A\cap B \in \F$,
    \item $\forall A \subseteq B \subseteq X: A \in \F \Rightarrow B \in \F$.
  \end{enumerate}
  Denote by ${\rm Flt}(X)$ the set of all filters on $X$.
  Let $\F'$ be another filter on $X$. If $\F \subseteq \F'$ we call $\F'$ finer than $\F$ or $\F$ coarser than $\F'$. A family of subsets $\mathcal{F}_0 \subseteq \mathcal{F}$ of $X$ is called a \textbf{filter basis} of $\mathcal{F}$ if
  \begin{equation*}
    \forall F \in \mathcal{F} \exists F_0 \in \mathcal{F}_0\colon F_0 \subseteq F. \qedhere
  \end{equation*}
\end{defin}

\begin{defin}\label{defin:nbs}
  Let $X$ be a topological space. A \textbf{neighborhood basis} at $x \in X$ is a filter basis of the neighborhood filter $\mathcal{N}_X(x)$.
\end{defin}
  
\begin{lemma}\label{lem:filbas}
  Let $\mathcal{B}$ be a collection of non-empty subsets of a non-empty set $X$. $\mathcal{B}$ is a basis of a filter if and only if
  \begin{equation}\label{eq:filterbasis}
    \forall B_1, B_2 \in \mathcal{B}\exists B_3 \in \mathcal{B}: B_3 \subseteq B_1 \cap B_2.
  \end{equation}
\end{lemma}

\begin{proof}
  Assume $\mathcal{B}$ is the basis of a filter $\mathcal{F}$ on $X$. It holds especially that $\mathcal{B} \subseteq \mathcal{F}$. Let $B_1, B_2 \in \mathcal{B}$. Since these are also filter elements it follows that $B_1 \cap B_2 \in \mathcal{B}$ which proves the claim.

  Now assume that $\mathcal{B}$ fulfills equations \ref{eq:filterbasis} and let $B_1, B_2 \in \mathcal{B}$. Define the filter \[\mathcal{F} := \{ F \in \PowS(X)\setminus\{ \emptyset \}\colon \exists B\in\mathcal{B}\colon B \subseteq F \}.\] It is clear that $\emptyset \notin \mathcal{F} \ni X$. Furthermore this filter is closed under the operation of taking supersets by definition. Now suppose that $F_1, F_2 \in \mathcal{F}$. There exist $B_1, B_2 \in \mathcal{B}$ with $B_1 \subseteq F_1$ and $B_2 \subseteq F_2$. Now by Equation \ref{eq:filterbasis} it follows that there is $B_3 \in \mathcal{B}$ with $B_3 \subseteq B_1 \cap B_2$ and thus
  \begin{equation*}
    \emptyset \neq B_3 \subseteq B_1 \cap B_2 \subseteq F_1 \cap F_2.
  \end{equation*}
  Hence $F_1 \cap F_2 \in \mathcal{F}$.
\end{proof}

From this point the notation $\mathcal{F}(\mathcal{B}) := \{F \in \PowS(X) \colon \exists B\in\mathcal{B}\colon B\subseteq F\}$ represents the filter on $X$ generated by the filter basis $\mathcal{B}$.

\begin{col}\label{cor:filbas}
  Let $X$ be a non-empty set, $A \subseteq X$ and let $\mathcal{F}$ be a filter on $X$. If $A \cap F \neq \emptyset$ for all $F\in\mathcal{F}$ then the set $\mathcal{A} := \{A \cap F\colon F\in \mathcal{F}\}$ forms a filter basis of a filter which is finer than $\mathcal{F}$.
\end{col}

\begin{proof}
  Let $F_1, F_2 \in \mathcal{F}$ and define $A_1 := F_1 \cap A$ and $A_2 := F_2 \cap A$. Now consider
  \begin{equation*}
    A_1 \cap A_2 = (F_1 \cap A) \cap (F_2 \cap A) = F_1 \cap F_2 \cap A = F_3 \cap A \in \mathcal{A}
  \end{equation*}
  where $F_3 = F_1 \cap F_2 \in \mathcal{F}$. Thus $\mathcal{A}$ is a filter basis by Lemma \ref{lem:filbas}.

  Now consider the filter $\mathcal{F}(\mathcal{A})$. Since $\mathcal{A}$ is a filter basis of this filter all elements of the form $F \cap A$ for $F\in\mathcal{F}$ are contained in $\mathcal{F}(\mathcal{A})$. For each $F\in\mathcal{F}$ there exists $F\cap A\in \mathcal{F}(\mathcal{A})$ and since $F\cap A \subseteq F$ it follows that $F\in \mathcal{F}(\mathcal{A})$. 
\end{proof}

\begin{defin} 
  An \textbf{ultrafilter} $\F$ on $X$ is a filter on $X$ with the property that if there is another filter on X called $\F'$ such that $\F \subseteq \F'$ it follows that $\F = \F'$. Denote by ${\rm UFlt}(X)$ the set of all ultrafilters on $X$.
  If $\bigcap \F = \emptyset$ the ultrafilter is called \textbf{free}.
\end{defin}

\begin{lemma}\label{lem:ufltlemma}
  Let $X$ be a non-empty set and let $\mathcal{F} \in {\rm UFlt(X)}$ then
  \begin{equation*}
    \forall A \subseteq X\colon A \in \mathcal{F} \lor A^c \in \mathcal{F}.
  \end{equation*}
\end{lemma}

\begin{proof}
  Since $A \cap A^c = \emptyset$ there are no two sets $F_1, F_2 \in \mathcal{F}$ with $F_1 \subseteq A$ and $F_2 \subseteq A^c$. This means that for all $F \in \mathcal{F}$ either $F \cap A \neq \emptyset$ or $F \cap A^c \neq \emptyset$. Assume w.l.o.g. that $A \cap F \neq \emptyset$ for all $F \in \mathcal{F}$. From this it follows that $\{F \cap A\colon F\in \mathcal{F}\}$ is a filter basis of a filter $\mathcal{G}$ which is finer than $\mathcal{F}$ by Corollary \ref{cor:filbas}. Since $\mathcal{F}$ is an ultrafilter it follows that $\mathcal{F} = \mathcal{G}$ and thus $A \in \mathcal{F}$. \cite[5.12 Satz]{BvQMT}
\end{proof}

\begin{defin}
  Let $X$ be a topological space. A filter $\F$ converges to an element $x$ in $X$ ($\F \to x$) if $\mathcal{N}_x \subseteq \F$. 
  
  Let $Y$ be another topological space and $f: X \to Y$ a continuous function. 
  We call $f(\F)$ the \textbf{image filter} of $\F$ under $f$ which is the filter with filter basis $\{f(F)\colon F \in \F\}$. Another notation for the convergence of a filter is $\lim\limits_{F\to\F}f(F)$ (instead of $f(\F) \to y$).
\end{defin}

Indeed the new definition of convergence extends the old one from \ref{def:convtop}. Let $X$ be a topological space and let $(x_n)_{n\in\N}$ be a sequence in $X$ which converges to $x \in X$. Then there exists a filter on $X$ which converges to $x$ namely $\mathcal{N}_X(x)$. The inverse is not true because the new definition of convergence now allows the convergence to the point $\omega_1$ in Example \ref{ex:ordertop-uncount}. And it is also true that if $A \subseteq X$ and $a \in \bar{A}$ then there exists a filter that converges to $a$. This is not true for sequences by Example \ref{ex:ordertop-uncount}.

\begin{thm}\label{thm:fcl}
  Let $X$ be a topological space and let $A \subseteq X$. Then
  \begin{equation*}
    x \in \bar{A} \: \iff \: \exists \F \in {\rm Flt}(X): A \in \F \: \land \: \F \to x.
  \end{equation*}
\end{thm}

\begin{proof}
  See \cite[5.17 Satz]{BvQMT}.
\end{proof}

\begin{defin}
  The \textbf{Cofinite-Filter} $\F_{CF}$ on an infinite set $X$ is defined as 
  \begin{equation*}
    \F_{CF} := \{ F \in \P(X)\colon \left| F^c \right| < \infty\}.\qedhere
  \end{equation*}
\end{defin}

\begin{lemma}\label{lem:coffil}
  Let $X$ be an infinite set, then the Cofinite-Filter on $X$ is a free filter.
\end{lemma}

\begin{proof}
  It is trivial to see that $X \in \F_{CF}\:$ and $\:\emptyset \notin \F_{CF}$. So let $A, B \in \F_{CF}$ then
  \begin{align*}
    (A \cap B)^c = A^c \cup B^c \Rightarrow \left| (A \cap B)^c \right| = \left | A^c \cup B^c \right| \leq \left| A^c \right| + \left| B^c \right| < \infty
  \end{align*}
  since $A^c$ and $B^c$ are finite.
  Now assume that $A \subseteq B$. It follows that
  \begin{align*}
    A \subseteq B \: \Rightarrow \: B^c \subseteq A^c \: \Rightarrow \left|B^c\right| \leq \left|A^c\right| < \infty.
  \end{align*}
  This proves that $\F_{CF}$ is a filter on $X$.
  Now assume that $\bigcap \F_{CF}$ is non-empty and let $x \in \bigcap \F_{CF}$. Now let $F \in \F_{CF}$. By assumption it follows that $x \in F$ and
  \begin{align*}
    \left|(F\setminus \{x\})^c\right| = \left| F^c \right| + \left| \{x \} \right| = \left| F^c \right| + 1 < \infty
  \end{align*}
  and hence $F\setminus \{x\} \in \F_{CF}$. $\lightning$
\end{proof}

\begin{thm}\label{thm:ulfil}
  Let $X$ be non-empty set. Every filter on X is contained in an ultrafilter on $X$.
\end{thm}

\begin{proof}
  Let $\F$ be a filter on $X$ and let $\Phi$ be the set of all filters on $X$ that are finer than $\F$. This set forms a partial order together with the $\subseteq$ relation.
  If $\Phi_1$ is a totally ordered subset of $\Phi$ then define $\F' := \bigcup\limits_{\mathcal{G} \in \Phi} \mathcal{G}$.
  \textbf{Claim: } $\F'$ is a filter on $X$. The first property is clear because every filter contains $X$ and does not contain $\emptyset$.
  Now let $A, B \in \F'$. This means that there are filters $\mathcal{G}_1$ and $\mathcal{G}_2$ in $\F'$ with $A \in \mathcal{G}_1$ and $B \in \mathcal{G}_2$.
  Since $\mathcal{G}_1 \in \Phi_1$ and $\mathcal{G}_2 \in \Phi_1$ it follows that $\mathcal{G}_1 \subseteq \mathcal{G}_2$ or $\mathcal{G}_2 \subseteq \mathcal{G}_1$. Suppose w.l.o.g. that $\mathcal{G}_1 \subseteq \mathcal{G}_2$.
  It follows that
  \begin{equation*}
    A \in \mathcal{G}_2 \Rightarrow A \cup B \in \mathcal{G}_2 \Rightarrow A \cup B \in \F'.
  \end{equation*}
  Now suppose that $A \in \F'$ and $A \subseteq B \subseteq X$. Since there is an $\mathcal{G} \in \Phi_1$ with $A \in \mathcal{G}$ we can conclude that
  \begin{equation*}
    B \in \mathcal{G} \Rightarrow B \in \F'.
  \end{equation*}
  This proves the claim. $\F'$ is an upper bound of $\Phi_1$ and thus $\Phi_1$ is inductively sorted. The existence of an ultrafilter on $X$ follows from Zorn's Lemma. \cite[5.12 Satz]{BvQMT}
\end{proof}

\begin{col}\label{col:exfreeuf}
  Let $X$ be a infinite set. There exists a free ultrafilter on $X$.
\end{col}

\begin{proof}
  Follows directly from \ref{lem:coffil} and \ref{thm:ulfil}.
\end{proof}

\begin{lemma}
  Every ultrafilter on a non-empty finite set $X$ converges to a point of $X$.
\end{lemma}

\begin{proof}
  In the case of a discrete set, the ultrafilter converges against an element if the singleton set containing this element is contained inside of the ultrafilter.
  Assume the cardinality of $X$ is 1. Then the claim follows directly by the definition of the ultrafilter.
Now assume that the Lemma is true for a set with cardinality $n \in \N$ and let $X$ be a set with cardinality $n+1$ and additionally let $\mathcal{F}$ be an ultrafilter on $X$. Write $X = \{1, \ldots, n\} \cup \{n+1\}$. By Theorem \ref{lem:ufltlemma} either $\{n+1\} \in \mathcal{F}$ which ends the proof by the definition of convergence of an ultrafilter or $\{1,\ldots,n\} \in \mathcal{F}$ which ends the proof by the inductive assumption.
\end{proof}

\subsubsection{Uniform spaces}

\begin{defin}
  Let $X$ be a set and let $A, B \subseteq X \times X$ be realtions on $X$.
  Define the following other relations on $X$
  \begin{itemize}
    \item $A^{-1} := \{(a_2, a_1)\colon (a_1, a_2) \in A\}$,
    \item $A \circ B := \{(a, b)\colon \exists c \in X\colon (a, c) \in A \: \land \: (c, b) \in B\}$.
  \end{itemize}
  Additionally, define $A^2 := A \circ A$.  
\end{defin}

So far, we have discussed a generalized concept of convergence in general topological spaces. In the analysis of topological groups, we also need the a generalized concept of completness of a space in terms of filters. This means a general notion of \textit{cauchy sequences} is needed called \textit{cauchy filters}. It will also be possible to say that a space is complete similiar to the metric case using this new notion.

Firstly, we will define a structure that arises naturally in the analysis of topological groups and is also compatible with filters and cauchy filters.
\begin{defin}
  Let $X$ be a set. A non-empty subset $\mathcal{U} \subseteq \PowS(X\times X)$ is called a \textbf{uniform structure} on $X$ if
  \begin{enumerate}
    \item $\mathcal{U}$ is a filter,
    \item\label{def:unif3} $U\in\mathcal{U}\colon \{(x,x)\colon x\in X\} = \Delta_X \subseteq U$,
    \item\label{def:unif1} $U \in \mathcal{U}\colon U^{-1} \in \mathcal{U}$,
    \item\label{def:unif2} $U \in \mathcal{U}\exists V\in \mathcal{U}\colon V^2 \subseteq U$.
  \end{enumerate}
  The elements of the uniform structure are called \textbf{entourages}. Let $E \in \mathcal{U}$ and let $x, y \in X$. The points $x$ and $y$ are called $E$-close if $(x, y) \in E$. Similiarly a subset $A \subseteq X$ is called $E$-small if $A\times A \subseteq E$.
  
  Let $X$ and $Y$ be uniform spaces with uniform structures $\mathcal{U}_X$ and $\mathcal{U}_Y$. A map $f\colon X \to Y$ is called \textbf{uniformly continuous} if for each $W \in \mathcal{U}_Y$ there is $V \in \mathcal{U}_X$ such that $$(f\times f)(V) \subseteq W$$ where $$(f\times f) \colon X\times X \to Y \times Y, (x_1, x_2) \mapsto (f(x_1), f(x_2)).$$

  A set $\mathcal{B} \subseteq X\times X$ is called \textbf{fundamental system of neighborhoods} of the uniform structure of $X$ if every entourage $E$ of $X$ contains a set $B \in \mathcal{B}$. A non-empty fundamental system of neighborhoods is a filter basis for a uniform structure on $X$.

  Let $A \subseteq X$ be a subset of $X$. Then the set $$\mathcal{U}_A := \{E \cap (A\times A)\colon E\in \mathcal{U}_X\}$$ is a uniform structure on $A$ induced by the uniform structure on $X$.  
\end{defin}
The definition of uniform structures generalize metric spaces in a similiar sense that filter convergence generalizes convergence in metric spaces to a broader range of topological spaces. The axiom \ref{def:unif3} of a uniform space reflects the fact that every point should be close to itself with respect to every entourage. This is similiar to the metric case, where the distance from a point to itself should be 0. In the case of uniform spaces, this condition is more general because we can have uniform spaces that are not Hausdorff. Axiom \ref{def:unif1} of a uniform space is the equivalent to the axiom of symmetry a metric has to fulfill and axiom \ref{def:unif2} is the equivalent to the triangle inequality.  

A uniform space comes naturally equipped with a topology, that is induced by the uniformity of the space. This can easily be seen when considering a uniform space $X$ with uniform structure $\mathcal{U}$. Then we can define a system of neighborhoods $$ \mathcal{N}_X(x) := \{V(x)\colon V \in \mathcal{U}\}, \: V(x) := \{ (x,y) \in X \times X\colon (x,y) \in V\} $$ for each point $x \in X$. The inverse is not true in general. But if the uniformity induces a topology, this topology is unique.

\begin{thm}
  Let $X$ be a uniform space with uniformity $\mathcal{U}$. Then this uniformity induces a unique topology on $X$.
\end{thm}

\begin{proof}
  See \cite[Satz 11.5]{BvQMT}.
\end{proof}

It is important to notice that there is not a one-to-one relation between uniform structures and the topologies induced by them. There can be multiple different uniformities that induce the same topology on the underlying space. This is especially true for topological groups which are equipped with three uniform structures that arise in a canonical way from the group topology and the group operation but all induce the same topology, namely the group topology.

\begin{definthm}
  Let $X$ be a compact topological space. Then the set $\mathcal{E}(X)$ defined as $$ \mathcal{E}(X) := \mathcal{N}_{X\times X}(\Delta_X)$$ is a uniform structure on $X$ which naturally arises from the topology on $X$.
\end{definthm}

\begin{proof}
  See \cite[p. 199f.]{bour1998}.
\end{proof}

\begin{defin}
  Let $X$ be a uniform space with uniformity $\mathcal{U}$. A filter $\mathcal{F} \in {\rm Flt}(X)$ is called \textbf{cauchy} with respect to the uniform structure $\mathcal{U}$ if for each entourage $U \in \mathcal{U}$ there exists an element $F \in \mathcal{F}$ such that $F$ is $U$-small, i.e., $F \times F \subseteq U$.

  The uniform space $X$ is called \textbf{Raikov-complete} (or \textbf{complete}) if every cauchy filter in $X$ converges to a point in $X$.
\end{defin}

\begin{ex}\label{ex:cxycomp}
  Let $X$ be a topological space and let $Y$ be uniform space with uniform structure $\mathcal{U}_Y$. Define for every $V \in \mathcal{U}_Y$ the following set $$ W(V) = \{(f,g) \in C(X,Y)\times C(X,Y)\colon \forall x\in X\colon (f(x),g(x)) \in V\}.$$ Then the set $\{W(V)\colon V\in \mathcal{U}_Y\}$ is a fundamental system of neighborhoods for the uniform structure of uniform convergence $C_u(X,Y)$ on the set $C(X, Y)$ of all continuous functions form $X$ to $Y$. This space is Raikov-complete if $Y$ is Raikov-complete. 
\end{ex}

\begin{proof}
  See \cite[p.183ff.]{BvQMT}.
\end{proof}

\begin{lemma}
  Let $X$ and $Y$ be uniform spaces and let $f\colon X \to Y$ be a uniformly continuous map between them. If $\mathcal{F} \in {\rm Flt}(X)$ is cauchy in $X$ then the image filter $f(\mathcal{F})$ is cauchy in $Y$.
\end{lemma}

\begin{proof}
  Let $E'$ be a entourage of $Y$ and define $E := (f\times f)^{-1}(E')$ which is an entourage of $X$. Since $\mathcal{F}$ is cauchy in $X$ there exists a filter element $F \in \mathcal{F}$ such that $F \times F \subseteq E$. But now it follows that $$ F \times F \subseteq E = (f \times f)^{-1}(E') \Rightarrow (f\times f)(F\times F) \subseteq (f\times f)(E) \subseteq E'.$$ We can conclude that $f(\mathcal{F})$ is cauchy in $Y$ because $f(F) \in f(\mathcal{F})$.
\end{proof}

\subsection{Topological Algebra}

\begin{defin}
  Let $G$ be a group with group operation $+_G$ and $\T$ a topology on the underlying set of $G$. Additionally define ${\rm mul}\colon G \times G \to G, (g, h) \mapsto g \cdot_G h$ as the multiplication map of $G$ and ${\rm inv}\colon G \to G, g \mapsto g^{-1}$ as the inverse map of $G$. Then $G$ is called a \textbf{topological group} if both ${\rm mul}$ and ${\rm inv}$ are continuous maps with respect to $\T$.
  Let $X$ be a set. An action of $G$ on $X$ is a map $\lambda\colon G\times X \to X, (g, x) \mapsto \lambda(g, x) =: g \anddot x$ that fulfills the following properties
  \begin{enumerate}
    \item $\forall g, h \in G: h \anddot (g \anddot x) = (h \cdot_G g) \anddot x$,
    \item $\lambda(e_G, x) = x$ for all $x \in X$.
  \end{enumerate}
  If $X$ is a topological space, then the action of $G$ on $X$ is continuous if $\lambda$ is a continuous map.
\end{defin}

\begin{ex}
  Let $X$ be a compact, Hausdorff topological space. The set $\Homeo(X)$ of all homeomorphism from $X$ to $X$ together with the operation of composition of functions and taking inverses of functions is a group. It becomes a topological group when equipped with the compact-open topology.
\end{ex}

\begin{proof}
  Take $f,g \in \Homeo(X)$. Since the composition of two continuous functions is continuous (see \cite[Satz 2.20]{BvQMT}) and composition of two bijections is again a bijection, $f\circ g$ is a continuous bijection. And since the inverses of $f$ and $g$ are continuous it is even a homeomorphism, which means that $f\circ g \in \Homeo(X)$. Since $\id_X \in \Homeo(X)$ it follows that
  \begin{equation*}
    \id_X \circ f = f = f \circ \id_X.
  \end{equation*}
  This means that $\id_X$ is the neutral element of the group. Lastely it is clear that $f \circ f^{-1} = \id_X$ and $f^{-1} \circ f = \id_X$, so every element has an inverse.
Next we have to show that $\circ$ and taking function inverses are continuous maps. To this end assume that $W(C, W)$ is an open neighborhood of $g\circ f$. Firstly we now that $f(C) \subseteq g^{-1}(W)$ and that $f(C)$ is a compact set. We can choose for each $x \in f(C)$ a compact neighborhood $V_x$ of $x$ with $V_x \subseteq g^{-1}(W)$ since $X$ is compact and thus locally-compact. Since $f(C)$ is compact there exists a finite set $F \subseteq f(C)$ such that
\begin{equation*}
  f(C) \subseteq \bigcup\limits_{f\in F}\overset{\circ}{V_f} =: W'.
\end{equation*}
Also define the set $C' := \bigcup\limits_{f\in F} V_f$ which is compact.

So we can conclude that $f \in W(C, W')$ and $g \in W(C', W)$. And since $W' \subseteq C'$ it follows for all $f' \in W(C, W')$ and $g \in W(C', W)$ that
\begin{equation*}
  g' \circ f' \in W(C, W),
\end{equation*}
and hence $W(C, W') \circ W(C', W) \subseteq W(C, W)$. It follows that $\circ$ is continuous.

  And it is also clear by definition that $f^{-1} \in \Homeo(X)$ by definition of a homeomorphism.
  Let $W(K,U)$ be an open neighborhood of $f^{-1}$ with $K \subseteq X$ compact and $U \subseteq X$ open. Define the closed set $A := U^c$ and the open set $V := K^c$. Since $A$ is closed in $X$ it follows that $A$ is also compact in $X$. Also note that $K\cap f(A) = \emptyset$ since $K \subseteq f(U)$. 

  Now it follows that
  \begin{equation*}
    f^{-1}(K)  \subseteq U \iff K \subseteq f(U) \iff f(U)^c \subseteq K^c \iff f(A) \subseteq V, 
  \end{equation*}
  hence $f\in W(A, V)$. Furthermore, for each $g\in W(A, V)$ it holds that
  \begin{equation*}
    g(A) \subseteq V \iff V^c \subseteq g(A)^c \iff K \subseteq g(U) \iff g^{-1}(K) \subseteq U
  \end{equation*}
  and thus we get $g^{-1} \in W(K, U)$.
  We can conclude that $W(A, V)^{-1} \subseteq W(K, U)$ which means that taking inverses is continuous.
\end{proof}

\begin{thm}\label{thm:homeocomp}
  Let $X$ be a compact, Hausdorff topological space. The topological group $\Homeo(X)$ is Raikov-complete with respect to the two-sided uniformity $\mathfrak{S}$ of $\Homeo(X)$. 
\end{thm}

\begin{proof}
  Let $\mathcal{F} \in {\rm Flt}(X)$ be a cauchy filter with respect to $\mathfrak{S}$. Since the inverse map is uniformly continuous, the filter $\mathcal{F}^{-1}$ defined as ${\rm inv}(\mathcal{F})$ is also a cauchy filter. Since the space $C(X, X)$ contains $\Homeo(X)$ and is complete (see \ref{ex:cxycomp}) the filter limits $g := \lim \mathcal{F}$ and $h := \lim \mathcal{F}^{-1}$ exist in $C(X,X)$. Now consider $$ g \circ h = (\lim \mathcal{F}) \circ (\lim \mathcal{F}^{-1}) = \lim {\mathcal{F} \circ \mathcal{F}^{-1}}. $$ The exchange of function composition and the limit of filters is possible since the function composition is a jointly continuous map.

  Let $E_0, E \in \mathcal{E}(X)$ such that $$ E^{-1} = E \quad \land \quad E \circ E \subseteq E_0.$$ Let $U := \{f \in \Homeo(X)\colon \forall x \in X: (g(x), f(x)) \in E\}$ which is a neighborhood of $g$ in $\Homeo(X)$. By the definition of filter convergence we know that there exists $F \in \mathcal{F}$ such that $F \subseteq U$. Now let $(f, \tilde{f}) \in F \times F$ and observe for all $x \in X$ that $$ \left(f(\tilde{f}^{-1}(x)), g(\tilde{f}^{-1}(x))\right) \in E \: \land \: \left(g(\tilde{f}^{-1}(x)), x\right) \in E \: \Rightarrow \: \left(f(\tilde{f}^{-1}(x)), x\right) \in E_0.$$ The fact that $\left(g(\tilde{f}^{-1}(x)), x\right) \in E$ for all $x \in X$ follows by $\left(g(x), \tilde{f}(x)\right) \in E$ for all $x \in X$ and the substitution $y = \tilde{f}^{-1}(x)$. We can conclude that $F \circ F \subseteq \tilde{E}_0({\rm id}_X)$. 

  From this is follows that $\lim \mathcal{F} \circ \mathcal{F}^{-1} = {\rm id}_X$ and thus $h = g^{-1}$, hence $g \in \Homeo(X)$.
\end{proof}


In every topological group there arise three uniform structures in a natural way that all induce the group topology.

\begin{defin}
  Let $G$ be a topological group. Define for each $V \in \mathcal{N}(G)$ the sets
  \begin{align*}
    R_V := \{(x, y) \in G\times G\colon xy^{-1} \in V\}, \\
    L_V := \{(x, y) \in G\times G\colon x^{-1}y \in V\}.
  \end{align*}
  Now we can define the \textbf{right uniform structure} $\mathcal{R}$ and the \textbf{left uniform structure} $\mathcal{L}$ in the following way
  \begin{align*}
    \mathfrak{R} := \{ E \subseteq G\times G \colon \exists V \in \mathcal{N}(G)\colon R_V \subseteq E \}, \\
    \mathfrak{L} := \{ E \subseteq G\times G \colon \exists V \in \mathcal{N}(G)\colon L_V \subseteq E \}.
  \end{align*}

  The two \textbf{sided uniform structure} $\mathfrak{S}$ on $G$ can now be defined as
  \begin{equation*}
    \mathfrak{S}:= \mathfrak{R} \cap \mathfrak{L}.\qedhere
  \end{equation*}
\end{defin}

\subsubsection{Extreme Amenability}

\begin{defin}
  A topological group $G$ is called \textbf{extremely amenable} if every continuous action of G on a non-empty, compact, Hausdorff space admits a fixed point.
\end{defin}

\begin{defin}
  Let $X$ and $Y$ be sets. A step function $f\colon X \to Y$ with finite range induces a finite partition on X called $\P_f$ in the following way
  \begin{equation*}
    \P_f := \{f^{-1}(\{y\})\colon y \in Y\}.
  \end{equation*}
  This is a well-defined finite partion because of the assumption that $f$ has finite range.
\end{defin}

\begin{lemma}
  Let $G$ be a topological group, then it holds that
  \vspace*{-7px}
  \begin{itemize}
    \item $\forall U\in \mathcal{N}(G)\exists V\in\mathcal{N}(G)\colon V\cdot V \subseteq U$,
  \vspace*{-7px}
    \item $\forall U\in \mathcal{N}(G)\exists V\in\mathcal{N}(G)\colon V^{-1} \subseteq U$,
  \vspace*{-7px}
    \item $\forall U\in \mathcal{N}(G)\forall g\in G\exists V\in\mathcal{N}(G)\colon gVg^{-1} \subseteq U$.
  \end{itemize}
\end{lemma}

\begin{proof}
  See \cite[Satz 16.16]{BvQMT}.
\end{proof}

\begin{defin}\label{defin:sf}
  Let $X$ be a set and let $\mu\colon \mathcal{B} \to [0, \infty)$ be a submeasure on $X$ on the boolean algebra $\mathcal{B} \subseteq \PowS(X)$. Then the set
  \begin{equation*}
    S(\mu, G) := \left\{ f: X \to G\colon f \:\:\mu-\text{measurable} \: \land \: \left| f(X) \right| < \infty \right\}
  \end{equation*}
  for a topological group $G$ is the set of all measurable step functions with finite range on $X$ and values in $G$.
\end{defin}

\begin{thm}\label{thm:stop}
  Let $G$ be a topological group and let $\mu\colon \mathfrak{P}(X) \supseteq \mathcal{B} \to [0,\infty)$ be a submeasure on $X$. Then the set of simple functions $S(\mu, G)$ together with the operation of pointwise multiplication \[S(\mu, G) \times S(\mu, G) \to S(\mu, G), \: (f, g) \mapsto f \cdot g\] where $(f \cdot g)(x) = f(x) \cdot g(x)$ for $x \in X$ and pointwise inverses \[S(\mu, G) \to S(\mu, G), \: f \mapsto f^{-1}\] where $(f^{-1})(x) = (f(x))^{-1}$ for $x \in X$ is a topological group with respect to the topology of convergence in submeasure. A basis element of this topology is given by \[V_\varepsilon(f) = \{h\in S(\mu, G)\colon \mu(\left\{x\in X\colon h(x)\notin f(x)V \right\}) < \varepsilon\} \] where $f\in S(\mu, G)$, $\varepsilon > 0$ and $V \in \mathcal{N}(G)$.
\end{thm}

\begin{proof}
  The fact, that $S(\mu, G)$ together with the pointwise multiplication of maps forms a group follows trivially from the fact that $G$ is a group already. Hence, it remains to show that the maps ${\rm mul}$ and ${\rm inv}$ are continuous with respect to the group topology.
 
  Let $\varepsilon \in \mathbb{R}_{>0}$ and let $V \in \mathcal{N}(G)$. Let $(g,h) \in S(\mu, G) \times S(\mu, G)$ and consider $U := {\rm mul}^{-1}(V_\varepsilon({\rm mul}(g,h)))$. Choose $\tilde{V} \in \mathcal{N}(G)$ such that $\tilde{V}^2 \subseteq V$. Next define \[W := \left(\bigcap\limits_{x\in h(X)} x^{-1}V_xx \right) \cap \tilde{V}\] with $V_x\in \mathcal{N}(G)$ such that $x^{-1}V_xx \subseteq \tilde{V}$ for each $x \in h(X)$. Since the range of $h$ is finite, $W$ is an open set containing the identity of $G$. Let $(\tilde{g}, \tilde{h})\in W_{\varepsilon\over 2}(g) \times W_{\varepsilon \over 2}(h)$. Now it follows that
  \begin{align*}
    \mu\left(\left\{x\in X\colon \tilde{g}(x) \notin g(x)W\right\}\right) < {\varepsilon\over 2} \: \land \: 
    \mu\left(\left\{x\in X\colon \tilde{h}(x) \notin h(x)W\right\}\right) < {\varepsilon \over 2}
  \end{align*}
  and 
  \begin{align*}
    &\mu\left(\left\{x \in X\colon \tilde{g}(x)\tilde{h}(x) \notin g(x)Wh(x)W \right\}\right) \\
    = \:&\mu\left(\left\{x \in X\colon \tilde{g}(x)\tilde{h}(x) \notin g(x)h(x)(h(x))^{-1}Wh(x)W \right\}\right) \\
    \geq \:&\mu\left(\left\{x \in X\colon \tilde{g}(x)\tilde{h}(x) \notin g(x)h(x)\tilde{V}W \right\}\right) \\
    \geq \:&\mu\left(\left\{x \in X\colon \tilde{g}(x)\tilde{h}(x) \notin g(x)h(x)\tilde{V}^2 \right\}\right) \\
    \geq \:&\mu\left(\left\{x \in X\colon \tilde{g}(x)\tilde{h}(x) \notin g(x)h(x)V \right\}\right) 
  \end{align*}
  and since
  \begin{align*}
    \mu\left(\left\{x \in X\colon \tilde{g}(x)\tilde{h}(x) \notin g(x)Wh(x)W \right\}\right) < {\varepsilon\over 2} + {\varepsilon\over 2} = \varepsilon
  \end{align*}
  we get that
  \begin{align*}
    \mu\left(\left\{x \in X\colon \tilde{g}(x)\tilde{h}(x) \notin g(x)h(x)V \right\}\right) < \varepsilon. 
  \end{align*}
  Finally we can conclude that ${\rm mul}(W_{\varepsilon \over 2}(g) \times W_{\varepsilon \over 2}(h)) \subseteq U$.

  Next consider the set $Q$ defined as \[ Q = \bigcap\limits_{y \in f(X)}yQ_yy^{-1}\] with $Q_y \in \mathcal{N}(G)$ and $yQ_yy^{-1} \subseteq V$ for each $y \in f(X)$. Since the range of $f$ is finite, $Q$ is a open set containing the identity of $G$.
  It follows that
  \begin{align*}
    \tilde{f} \in {\rm inv}(Q_\varepsilon(f)) \iff \varepsilon > \: &\mu(\{x \in X\colon (\tilde{f}(x))^{-1} \notin f(x)Q\}) \\
    = \: &\mu(\{x \in X\colon \tilde{f}(x) \notin Q^{-1}(f(x))^{-1}\})\\
    = \: &\mu(\{x \in X\colon \tilde{f}(x) \notin (f(x))^{-1}f(x)Q^{-1}(f(x))^{-1}\})\\
    \geq \: &\mu(\{x \in X\colon \tilde{f}(x) \notin (f(x))^{-1}V\})             
  \end{align*}
  which means that $\tilde{f} \in V_{\varepsilon}({\rm inv}(f))$ and thus ${\rm inv}(Q_\varepsilon(f)) \subseteq V_{\varepsilon}({\rm inv}(f))$.
\end{proof}

\begin{lemma}\label{lem:contextrai}
  Let $H$ be a dense subgroup of a topological group $G$ and let $f\colon H \to K$ be a homomorphism of $H$ to a Raikov-complete topological group $K$, then there exists an extension of $f$ to a continuous homomorphism $\hat{f}\colon G \to K$.
\end{lemma}

\begin{proof}
  See \cite[Proposition 3.6.12]{atop2008}.
\end{proof}

\begin{thm}
  Let $G$ be a topological group, let $H \subseteq G$ be a dense subgroup and let $X$ be an arbitrary compact, Hausdorff topological space.
  Then every continuous action $H \curvearrowright X$ can be extended to a continuous action $G \curvearrowright X$.
\end{thm}

\begin{proof}
  Let $\lambda\colon H\times X \to X$ be a continuous action of $H$ on $X$. Define the map $\varphi\colon H \to \Homeo(X)$ as $$ \varphi\colon H \to \Homeo(X), \: h \mapsto \left( x \mapsto \lambda(h, x) \right). $$ This map is well-defined since it is easy to show that every map $\lambda_h = \lambda(h, \cdot)\colon X\to X$ is a homeomorphism for each $h \in H$.

  \textbf{Claim:} The map $\varphi$ is a continuous homomorphism of topological groups.
  The fact that $\varphi$ is a homomorphism of groups trivially follows from the fact of $\lambda$ being a group action. It remains to be shown that the map is continuous. To this end let $K \subseteq X$ be non-empty and compact and let $U \subseteq X$ be non-empty and open and define $V := \lambda^{-1}(U) \subseteq H \times X$. Let $h \in \varphi^{-1}(W(K,U))$. 

  Firstly notice that $$ \forall x \in K\colon (h, x) \in V $$ since $K$ is compact and since the map $\lambda$ is continuous by assumption the set $V$ is open in $H \times X$. This means we can choose $W_x \in \mathcal{N}_h(G)$ and $O_x \in \mathcal{T}_X$ such that $(h, x) \in W_x \times O_x \subseteq V$ for each $x \in K$. The sets $O_x$ are an open covering of $K$. Hence, we can find a finite subcover $(O_{x_i})_{i\in \{1,\ldots,n\}}$ with $n \in \mathbb{N}$.   
  Next define the set $$W := \bigcap\limits_{i = 1}^n W_{x_i}$$ which is an open neighborhood of $h$ in $H$ and define $$ O := \bigcup\limits_{i=1}^n O_{x_i}.$$ We know that by definition $$ \forall g \in W\forall i \in \{1,\ldots,n\}\colon \lambda(g, O_{x_i}) \subseteq U \Rightarrow \forall g\in W:\lambda(g,O) \subseteq U$$ and since $K \subseteq O$ it follows that $$\forall g\in W\colon \lambda(g, K) \subseteq U$$ and thus $h \in W \subseteq \varphi^{-1}(W(K,U))$. Hence, the claim is proven.

  Since $H$ is a dense subgroup of the group $G$ and by \ref{thm:homeocomp} the group $\Homeo(X)$ is Raikov-complete by \ref{lem:contextrai} there exists a continuous extension of $\varphi$ called $\hat{\varphi}\colon G \to \Homeo(X)$. 
  The last step is to prove, that the map $$\hat{\lambda}\colon G \times X \to X, \: (g, x) \mapsto \hat{\varphi}(g)(x)$$ is a continuous group action.
  The fact that this a group action follows trivially from the fact the $\hat{\varphi}$ is a homomorphism of groups.

  It remains to show that the group action $\hat{\lambda}$ is continuous.
  To this end let $(g,x) \in G\times X$, choose $U \in \mathcal{T}_x$ with $\hat{\lambda}(g, x) \in U$.

  Now consider $V := (\hat{\varphi}(g))^{-1}(U)$. This set is open since $\hat{\varphi}(g)$ is a homeomorphism and the set contains the point $x$ since $\hat{\varphi}(g)(x) = \lambda(g, x) \in U$. The set $K := \bar{V}$ is compact because it is a closed subset of a compact space. Now take the basis element $W(K, U)$ of the compact open topology on $\Homeo(X)$ and define $$W := \hat{\varphi}^{-1}(W(K,U)).$$ Again this set is open since $\hat{\varphi}$ is continuous and it contains $g$. Furthermore it is true that $$ \forall \tilde{g}\in W\colon \hat{\lambda}(\tilde{g}, K) = \hat{\varphi}(\tilde{g})(K) \subseteq U$$ and since $V \subseteq K$ thus $$\forall \tilde{g}\in W\colon \hat{\lambda}(\tilde{g}, V) = \hat{\varphi}(\tilde{g})(V) \subseteq U.$$ We can now conclude that $\hat{\lambda}(W \times V) \subseteq U$ and since $(g, x) \in W\times V \in \mathcal{T}_{G\times X}$ the map $\hat{\lambda}$ is continuous.
\end{proof}

\subsection{Algebraic Topology}
In the following section let $Y$ be a topological space and let $X$ be a metric space with metric $d_X$.

\begin{defin}
  Let $X$ and $Y$ be topological spaces. The continuous maps $f,g\colon X \to Y$ are called \textbf{homotopy equivalent} or \textbf{homotopic} if there exists a continuous map $F\colon X \times I \to Y$ where $I = [0, 1]$ with the following property \[F(0,x) = f(x) \quad \text{and} \quad F(1,x) = g(x).\]
 Define the relation $\simeq$ on $C(X, Y)$ as \[f \simeq g \colon\iff f \text{ and } g \text{ are homotopy equivalent}.\] This relation is an equivalence relation (see \cite[Lemma 51.1]{MunTop}). 

  The continuous function $f$ is called a \textbf{homotopy equivalence} if there exists a $h \in C(Y, X)$ such that
  \begin{equation*}
    h \circ f \simeq \id_X \: \land \: f \circ h \simeq \id_Y.
  \end{equation*}
  The two spaces $X$ and $Y$ are \textbf{homotopy equivalent} ($X \simeq Y$) if there exists a homotopy equivalence $h \in C(X, Y)$ between them.

  A space is called \textbf{contractible} if it is homotopy equivalent to a one point space.
\end{defin}

\begin{lemma}
  Let $X$ and $Y$ be topological spaces, then
  \begin{equation*}
    X \cong Y \: \Rightarrow \: X \simeq Y.
  \end{equation*}
\end{lemma}

\begin{proof}
  Let $\psi\colon X \to Y$ be a homeomorphism. Then it follows that $\psi \circ \psi^{-1} \simeq \id_Y$ and $\psi^{–1} \circ \psi \simeq \id_X$.  
\end{proof}
