\section{Topology, Topological Algebra and Algebraic Topology}
\subsection{Topology}

\subsubsection{Topological Spaces}
\begin{defin}
  Let $X$ be a set. A \textbf{topology} $\T$ on $X$ is a collection of subsets of $X$ obeying the follwing axioms
  \begin{enumerate}
    \item $X \in \T \ni \emptyset$,
    \item $\bigcup\limits_{\alpha \in I}A_{\alpha} \in \T$, where $I$ is an abitrary index set and $A_{\alpha} \in \T$ for all $\alpha \in I$,
    \item $\bigcap\limits_{i=0}^n A_i$ for $n \in \N$ and $A_i \in \T$.
  \end{enumerate}
  
  The tuple $(X, \T_X)$ where $T_X$ is a topology on $X$, is called a \textbf{topological space}.
\end{defin}

In general it is said that $X$ is a topological space without mentioning the explicit topology. In these cases $\T_X$ will be the topology of the space $X$.

% TODO: basis, subbasis, generated topology
\begin{defin}
  
\end{defin}

\begin{ex}\label{ex:top}
  Here are some examples of topological spaces which will be needed later.
\begin{enumerate}[label=\textbf{\arabic*.}]
  \item \label{ex:top-chaotic} Let $X$ be a nonempty set. Then the set $\T_X := \{X, \emptyset\}$ is a topology on $X$ called the \textbf{chaotic topology}.
  \item \label{ex:top-order} Let $(P, <)$ be a totally ordered set\footnote{see \cite[p. 24]{MunTop} for definition of total order (there simple order)}. Define for $a, b \in P$ with $a < b$ the \textbf{open interval} from a to b as $(a,b) := \{ x \in X\colon a < x \: \land \: x < b \}$ and similiarly define the half open intervals $[a,b) := \{a\} \cup (a,b)$ and $(a,b] := (a,b) \cup \{ b \}$. 
  Then the \textbf{order topology} on $P$ is generated by the basis
  \begin{equation*}
    \mathcal{B} := \{ (a,b) \colon a, b \in P \} \: \cup \: \{[a_0, b)\colon b \in P\} \: \cup \: \{ (b, a_1]\colon b \in P\}
  \end{equation*}
  where $a_0$ is the smallest element in $P$ if existent and $a_1$ is the largest element in $P$ if existent.
\end{enumerate}
\end{ex}

\begin{defin}
  Let $X$ and $Y$ be topological spaces. A map $f\colon X \to Y$ is called \textbf{continuous} if for all $A \in \T_Y$ it follows that $f^{-1}(A) \in \T_X$.  
\end{defin}

\begin{defin}
  Let $X$ be a topological space and let $A \subseteq X$. A cover of $A$ is a collection of subsets $\left\{A_i \in \P(X)\colon i \in I \right\}$ with index set $I$ which has the following property
  \begin{equation}\label{eq:cover}
    \bigcup\limits_{i\in I}A_i = A.
  \end{equation}
  If the elements of the cover are open subsets of $X$ the cover is called an \textbf{open} cover.
  A subcollection of a cover of $A$ is called a \textbf{subcover} if it still fulfills equation \ref{eq:cover}.
\end{defin}

\begin{defin}
  Let $X$ be a topological space and $A \subseteq X$. $A$ is called \textbf{compact} if every open cover of $A$ has a finite subcover. 
\end{defin}

\subsubsection{Filters}
In the case of metric spaces $(M, d_M)$, a sequence $(x_i)_{i\in \N}$ is said to converge to a limit point $x \in M$ if $\forall \varepsilon>0\exists N \in \N\forall n \geq N: d_M(x, x_n) < \varepsilon$. This definition of convergence relies on very special topological properties of metric spaces and does not work in general topological spaces. There is a more general definition of sequences in topology namely \textbf{nets}, which generalize sequences by taking abitrary index sets so that uncountable sequences become possible. Working with nets has the same caveats as working with sequences (selection of a convergent subsequence in the case of a compact set for example). Fortunately there is yet another more general definition of convergence in topological spaces which will be used in this thesis and which does not have the downfalls of nets but still makes convergence of uncountable sequences possible.

\begin{defin}\label{def:convtop}
  Let $X$ be a topological space, let $x \in X$ and let $(x_i)_{i \in \N}$ be a sequence in $X$. Then the sequence of the $x_n$ converges against $x$ ($x_n \to x$) if
  \begin{equation*}
    \forall U \in \mathcal{N}_x\exists N\in\N\forall n \geq N\colon x_n \in U
  \end{equation*}
\end{defin}
This is a more general definition of convergence of sequences working in abitrary topological spaces. The very abstract definition of topological spaces makes this new notion of limits not as intuitiv as in the very special metric case. In a general topological space not every sequence that is convergent has a unique limit.

\begin{ex}
  Consider the topological space of the real numbers $\R$ but with the chaotic topology (see \ref{ex:top} \ref{ex:top-chaotic}). In this topological space, every sequence of numbers converges against every point. To this end consider a sequence $(x_i)_{i\in\N}$ in $\R$ and a real number $x$. We only have to check on open neighborhood of $x$ namely the set $\R$. And this trivially fulfills the condition of Definition \ref{def:convtop} because every sequence member is contained in it. So we have shown that an arbitrary sequence in this space converges against an arbitrary point.  
\end{ex}

This shows that additional assumptions have to be imposed onto a topological space such that the definition of converging against a limit point makes sense.

\begin{defin}
  A topological space $X$ is called \textbf{Hausdorff} if for any two points $x, y$ in $X$ there exists an open neighborhood $U$ of $x$ and an open neighborhood $V$ of $y$ which are disjoint.   
\end{defin}

This property is enough such that every convergent sequence has a unique limit.

\begin{thm}
  Let $X$ be a topological space. If $X$ is Hausdorff then every convergent sequence in $X$ has a unique limit.
\end{thm}

\begin{proof}
  Let $(a_n)_{n \in \N}$ be a sequence in $X$ and suppose it converges against $x$ and $y$ in $X$ with $x \neq y$. 
  Because of the assumption that $X$ is Hausdorff there are open neighborhoods of the limit points $U$ and $V$ which are disjoint. 
  Now we know that there is are natural numbers $n_0$ and $n_1$ with the property
  \begin{equation*}
    \forall n \geq n_0\colon a_n \in U \: \land \: \forall n \geq n_1\colon a_n \in V
  \end{equation*}
  This means for $n := \max(n_0, n_1)$ that $a_n \in U \cap V = \emptyset \:\: \lightning$.
\end{proof}

The next example will show that this definition is not general enough for every topological space.

% TODO: define order topology, basis, subbasis, accumulation point
\begin{ex}\label{ex:ordertop-uncount}
  Let $(\Omega, \geq)$ be an uncountable set which is well-ordered, has a biggest element $\omega_1$ and for all $\alpha \in \Omega$ with $\alpha < \omega_1$ the set $\{\beta \in \Omega\colon \beta \leq a\}$ is countable\footnote{This construction is possible because of the well-ordering pricinple. See \cite[p. 53]{BvQMT}}.
  Now define the topological space $\Omega$ with the order topology like in \ref{ex:top} \ref{ex:top-order} and $\Omega_0 := \Omega\setminus \{\omega_1\}$.
  It holds that $\omega_1$ is an accumulation point of $\Omega_0$ but there is no sequence in $\Omega_0$ that converges to $\omega_1$.
\end{ex}

\begin{proof}
  Suppose there is a sequence $(\alpha_n)_{n\in\N}$ in $\Omega_0$ such $\alpha_n \to \omega_1$. This means that $\sup_{n\in\N} a_n = \omega_1$. Now define
  \begin{equation*}
    A_n = \{\beta\colon \beta \leq a_n\}
  \end{equation*}
  for all $n \in \N$. Since the sets $A_n$ are all countable by definition of $\Omega$ the set
  \begin{equation*}
    B := \bigcup\limits_{n\in\N} A_n = \{\beta\colon \exists m\in\N \colon \beta \leq a_m \}
  \end{equation*}
  is also countable. This means the smallest element of $\Omega\setminus B$ is well-defined. Call it $\gamma$. Thus
  \begin{equation*}
    \beta \in B \iff \beta \leq \gamma. 
  \end{equation*}
  But by definition of $\Omega$ and the fact that $\gamma \in \Omega_0$ it follows that $\gamma < \omega_1$. Now we get
  \begin{equation*}
    \sup\limits_{n\in\N} a_n \leq \gamma < \omega_1. \:\: \lightning
  \end{equation*}
\end{proof}

This example illustrates the fact that not in every topological space, the standard definition of a sequence and convergence of a sequence is enough. There can be accumulation points that cannot be reached by any sequence. The problem is that the element $\omega_1$ has an uncountable neighborhood basis and sequences have only countable many elements such that the definition of convergence cannot be fulfilled by a sequence because of cardinality reasons. This problem cannot arise in metric spaces because the countability of all neighborhood basis is a condition for a topological space to be metrizable (see \cite[p. 130f]{MunTop}).

In the following a more general definition of convergence in a topological space is developed.
\begin{defin}
  A \textbf{filter} $\F$ on a set $X$ is a collection of subsets of $X$ such that:
  \begin{enumerate}
    \item $X \in \F$, $\emptyset \notin \F$,
    \item $\forall A, B \in \F\colon A\cap B \in \F$,
    \item $\forall A \subseteq B \subseteq X: A \in \F \Rightarrow B \in \F$.
  \end{enumerate}
\end{defin}

\begin{defin} 
  An \textbf{ultrafilter} $\F$ on $X$ is a filter on $X$ with the property that if there is another filter on X called $\F'$ such that $\F \subseteq \F'$ it follows that $\F = \F'$.
  If $\bigcap \F = \emptyset$ the ultrafilter is called \textbf{free}.
\end{defin}

\begin{defin}
  Let $X$ be a topological space. A filter $\F$ converges to an element $x$ in $X$ ($\F \to x$) if $\mathcal{N}_x \subseteq \F$. 
  
  Let $Y$ be another topological space and $f: X \to Y$ a continuous function. 
  We call $f(\F) := \{f(F)\colon F \in \F\}$ the \textbf{image filter} of $\F$ under $f$. Another notation for the convegence of a filter is $\lim\limits_{F\to\F}f(F)$ (instead of $f(\F) \to x$).
\end{defin}

Indeed the new definition of convergence extends the old one from \ref{def:convtop}. Let $X$ be a topological space and let $(x_n)_{n\in\N}$ be a sequence in $X$ which converges to $x \in X$. Then there exists a filter on $X$ which converges to $x$ namely $\mathcal{N}_x$. The inverse is of course not true because the new definition of convergence now allows the convergence to the point $\omega_1$ in the above example. And it is also true that if $A \subseteq X$ and $a \in \bar{A}$ then there exists a filter that converges to $a$. This is not true for sequences by example \ref{ex:ordertop-uncount}.

\begin{thm}
  % TODO: for all a \in \bar{A} there is a filter converging to $a$.
\end{thm}

\begin{defin}
  The \textbf{Cofinite-Filter} $\F_{CF}$ on a infinite set $X$ is defined as 
  \begin{equation*}
    \F_{CF} := \{ F \in \P(X)\colon \left| F^c \right| < \infty\}.
  \end{equation*}
\end{defin}

\begin{lemma}\label{lem:coffil}
  Let $X$ be an infinite set, then the Cofinite-Filter on $X$ is a free filter.
\end{lemma}

\begin{proof}
  It is trivial to see that $X \in \F_{CF}\:$ and $\:\emptyset \notin \F_{CF}$. So let $A, B \in \F_{CF}$ then
  \begin{align*}
    (A \cap B)^c = A^c \cup B^c \Rightarrow \left| (A \cap B)^c \right| = \left | A^c \cup B^c \right| \leq \left| A^c \right| + \left| B^c \right| < \infty
  \end{align*}
  since $A^c$ and $B^c$ are finite.
  Now assume that $A \subseteq B$. It follows that
  \begin{align*}
    A \subseteq B \: \Rightarrow \: B^c \subseteq A^c \: \Rightarrow \left|B^c\right| \leq \left|A^c\right| < \infty.
  \end{align*}
  This proofs that $\F_{CF}$ is a filter on $X$.
  Now assume that $\bigcap \F_{CF}$ is non-emtpy and let $x \in \bigcap \F_{CF}$. Now let $F \in \F_{CF}$. By assumption it follows that $x \in F$ and
  \begin{align*}
    \left|(F\setminus \{x\})^c\right| = \left| F^c \right| + \left| \{x \} \right| = \left| F^c \right| + 1 < \infty
  \end{align*}
  and hence $F\setminus \{x\} \in \F_{CF}$. $\lightning$
\end{proof}

\begin{thm}\label{thm:ulfil}
  Let $X$ be non-empty set. Every filter on X is contained in an ultrafiler on $X$.
\end{thm}

\begin{proof}
  
\end{proof}

\begin{col}
  Let $X$ be a infinite set. There exists a free ultrafilter on $X$.
\end{col}

\begin{proof}
  Follows directly from \ref{lem:coffil} and \ref{thm:ulfil}.
\end{proof}

\subsection{Topological Algebra}

\begin{defin}
  Let $G$ be a group with group operation $\cdot_G$ and $\T$ a topology on the undelying set of $G$. Additionally define $m\colon G \times G \to G, (g, h) \mapsto g \cdot_G h$ as the multiplication map of $G$ and $\iota\colon G \to G, g \mapsto g^{-1}$ as the inverse map of $G$. Then $G$ is called a \textbf{topological group} if both $m$ and $\iota$ are continous maps with respect to $\T$.
  Let $X$ be a set. An action of $G$ on $X$ is a map $\lambda\colon G\times X \to X, (g, x) \mapsto \lambda(g, x) =: g \anddot x$ if it fulfills the following properties
  \begin{enumerate}
    \item $\forall g, h \in G: h \anddot (g \anddot x) = (h \cdot g) \anddot x$,
    \item For neutral element $e_G \in G$: $\lambda(e, x) = x$ for all $x \in X$.
  \end{enumerate}
  If $X$ is a topological space, then the action of $G$ on $X$ is continuous if $\lambda$ is a continuous map.
\end{defin}

% TODO: explenation why this property is so important and hat it is used for
\begin{defin}
  A topological group $G$ is called \textbf{extremly ameanable} if each every continous action of G on a compact space has a fixed point.
\end{defin}

\subsection{Algebraic Topology}
