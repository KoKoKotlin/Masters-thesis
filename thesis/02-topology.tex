\section{Topology, Topological Algebra and Algebraic Topology}
\subsection{Topology}

\subsubsection{Topological Spaces}
\begin{defin}
  Let $X$ be a set. A \textbf{topology} $\T$ on $X$ is a collection of subsets of $X$ obeying the follwing axioms
  \begin{enumerate}
    \item $X \in \T \ni \emptyset$,
    \item $\bigcup\limits_{\alpha \in I}A_{\alpha} \in \T$, where $I$ is an abitrary index set and $A_{\alpha} \in \T$ for all $\alpha \in I$,
    \item $\bigcap\limits_{i=0}^n A_i$ for $n \in \N$ and $A_i \in \T$.
  \end{enumerate}
  
  The tuple $(X, \T_X)$ where $T_X$ is a topology on $X$, is called a \textbf{topological space}.
\end{defin}

In general it is said that $X$ is a topological space without mentioning the explicit topology. In these cases $\T_X$ will be the topology of the space $X$.

% TODO: basis, subbasis, generated topology
\begin{defin}
  
\end{defin}

\begin{ex}\label{ex:top}
  Here are some examples of topological spaces which will be needed later.
\begin{enumerate}[label=\textbf{\arabic*.}]
  \item \label{ex:top-chaotic} Let $X$ be a nonempty set. Then the set $\T_X := \{X, \emptyset\}$ is a topology on $X$ called the \textbf{chaotic topology}.
  \item \label{ex:top-order} Let $(P, <)$ be a totally ordered set\footnote{see \cite[p. 24]{MunTop} for definition of total order (there simple order)}. Define for $a, b \in P$ with $a < b$ the \textbf{open interval} from a to b as $(a,b) := \{ x \in X\colon a < x \: \land \: x < b \}$ and similiarly define the half open intervals $[a,b) := \{a\} \cup (a,b)$ and $(a,b] := (a,b) \cup \{ b \}$. 
  Then the \textbf{order topology} on $P$ is generated by the basis
  \begin{equation*}
    \mathcal{B} := \{ (a,b) \colon a, b \in P \} \: \cup \: \{[a_0, b)\colon b \in P\} \: \cup \: \{ (b, a_1]\colon b \in P\}
  \end{equation*}
  where $a_0$ is the smallest element in $P$ if existent and $a_1$ is the largest element in $P$ if existent.
\end{enumerate}
\end{ex}

\begin{defin}
  Let $X$ and $Y$ be topological spaces. A map $f\colon X \to Y$ is called \textbf{continuous} if for all $A \in \T_Y$ it follows that $f^{-1}(A) \in \T_X$.  
\end{defin}

\begin{defin}
  Let $X$ be a topological space and let $A \subseteq X$. A cover of $A$ is a collection of subsets $\left\{A_i \in \P(X)\colon i \in I \right\}$ with index set $I$ which has the following property
  \begin{equation}\label{eq:cover}
    \bigcup\limits_{i\in I}A_i = A.
  \end{equation}
  If the elements of the cover are open subsets of $X$ the cover is called an \textbf{open} cover.
  A subcollection of a cover of $A$ is called a \textbf{subcover} if it still fulfills equation \ref{eq:cover}.
\end{defin}

\begin{defin}
  Let $X$ be a topological space and $A \subseteq X$. $A$ is called \textbf{compact} if every open cover of $A$ has a finite subcover. 
\end{defin}

\subsubsection{Filters}
In the case of metric spaces $(M, d_M)$, a sequence $(x_i)_{i\in \N}$ is said to converge to a limit point $x \in M$ if $\forall \varepsilon>0\exists N \in \N\forall n \geq N: d_M(x, x_n) < \varepsilon$. This definition of convergence relies on very special topological properties of metric spaces and does not work in general topological spaces. There is a more general definition of sequences in topology namely \textbf{nets}, which generalize sequences by taking abitrary index sets so that uncountable sequences become possible. Working with nets has the same caveats as working with sequences (selection of a convergent subsequence in the case of a compact set for example). Fortunately there is yet another more general definition of convergence in topological spaces which will be used in this thesis and which does not have the downfalls of nets but still makes convergence of uncountable sequences possible.

\begin{defin}\label{def:convtop}
  Let $X$ be a topological space, let $x \in X$ and let $(x_i)_{i \in \N}$ be a sequence in $X$. Then the sequence of the $x_n$ converges against $x$ ($x_n \to x$) if
  \begin{equation*}
    \forall U \in \mathcal{N}_x\exists N\in\N\forall n \geq N\colon x_n \in U
  \end{equation*}
\end{defin}
This is a more general definition of convergence of sequences working in abitrary topological spaces. The very abstract definition of topological spaces makes this new notion of limits not as intuitiv as in the very special metric case. In a general topological space not every sequence that is convergent has a unique limit.

\begin{ex}
  Consider the topological space of the real numbers $\R$ but with the chaotic topology (see \ref{ex:top} \ref{ex:top-chaotic}). In this topological space, every sequence of numbers converges against every point. To this end consider a sequence $(x_i)_{i\in\N}$ in $\R$ and a real number $x$. We only have to check on open neighborhood of $x$ namely the set $\R$. And this trivially fulfills the condition of Definition \ref{def:convtop} because every sequence member is contained in it. So we have shown that an arbitrary sequence in this space converges against an arbitrary point.  
\end{ex}

This shows that additional assumptions have to be imposed onto a topological space such that the definition of converging against a limit point makes sense.

\begin{defin}
  A topological space $X$ is called \textbf{Hausdorff} if for any two points $x, y$ in $X$ there exists an open neighborhood $U$ of $x$ and an open neighborhood $V$ of $y$ which are disjoint.   
\end{defin}

This property is enough such that every convergent sequence has a unique limit.

\begin{thm}
  Let $X$ be a topological space. If $X$ is Hausdorff then every convergent sequence in $X$ has a unique limit.
\end{thm}

\begin{proof}
  Let $(a_n)_{n \in \N}$ be a sequence in $X$ and suppose it converges against $x$ and $y$ in $X$ with $x \neq y$. 
  Because of the assumption that $X$ is Hausdorff there are open neighborhoods of the limit points $U$ and $V$ which are disjoint. 
  Now we know that there is are natural numbers $n_0$ and $n_1$ with the property
  \begin{equation*}
    \forall n \geq n_0\colon a_n \in U \: \land \: \forall n \geq n_1\colon a_n \in V
  \end{equation*}
  This means for $n := \max(n_0, n_1)$ that $a_n \in U \cap V = \emptyset \:\: \lightning$.
\end{proof}

The next example will show that this definition is not general enough for every topological space.

% TODO: define order topology, basis, subbasis, accumulation point
\begin{ex}\label{ex:ordertop-uncount}
  Let $(\Omega, \geq)$ be an uncountable set which is well-ordered, has a biggest element $\omega_1$ and for all $\alpha \in \Omega$ with $\alpha < \omega_1$ the set $\{\beta \in \Omega\colon \beta \leq a\}$ is countable\footnote{This construction is possible because of the well-ordering pricinple. See \cite[p. 53]{BvQMT}}.
  Now define the topological space $\Omega$ with the order topology like in \ref{ex:top} \ref{ex:top-order} and $\Omega_0 := \Omega\setminus \{\omega_1\}$.
  It holds that $\omega_1$ is an accumulation point of $\Omega_0$ but there is no sequence in $\Omega_0$ that converges to $\omega_1$.
\end{ex}

\begin{proof}
  Suppose there is a sequence $(\alpha_n)_{n\in\N}$ in $\Omega_0$ such $\alpha_n \to \omega_1$. This means that $\sup_{n\in\N} a_n = \omega_1$. Now define
  \begin{equation*}
    A_n = \{\beta\colon \beta \leq a_n\}
  \end{equation*}
  for all $n \in \N$. Since the sets $A_n$ are all countable by definition of $\Omega$ the set
  \begin{equation*}
    B := \bigcup\limits_{n\in\N} A_n = \{\beta\colon \exists m\in\N \colon \beta \leq a_m \}
  \end{equation*}
  is also countable. This means the smallest element of $\Omega\setminus B$ is well-defined. Call it $\gamma$. Thus
  \begin{equation*}
    \beta \in B \iff \beta \leq \gamma. 
  \end{equation*}
  But by definition of $\Omega$ and the fact that $\gamma \in \Omega_0$ it follows that $\gamma < \omega_1$. Now we get
  \begin{equation*}
    \sup\limits_{n\in\N} a_n \leq \gamma < \omega_1. \:\: \lightning
  \end{equation*}
\end{proof}

This example illustrates the fact that not in every topological space, the standard definition of a sequence and convergence of a sequence is enough. There can be accumulation points that cannot be reached by any sequence. The problem is that the element $\omega_1$ has an uncountable neighborhood basis and sequences have only countable many elements such that the definition of convergence cannot be fulfilled by a sequence because of cardinality reasons. This problem cannot arise in metric spaces because the countability of all neighborhood basis is a condition for a topological space to be metrizable (see \cite[p. 130f]{MunTop}).

In the following a more general definition of convergence in a topological space is developed.
\begin{defin}
  A \textbf{filter} $\F$ on a set non-empty $X$ is a collection of subsets of $X$ such that:
  \begin{enumerate}
    \item $X \in \F$, $\emptyset \notin \F$,
    \item $\forall A, B \in \F\colon A\cap B \in \F$,
    \item $\forall A \subseteq B \subseteq X: A \in \F \Rightarrow B \in \F$.
  \end{enumerate}
  Let $\F'$ be another filter on $X$. If $\F \subseteq F'$ we call $\F'$ finer than $\F$ or $\F$ coarser then $\F'$. A family of subsets $\mathcal{F}_0 \subseteq \mathcal{F}$ of $X$ is called a \textbf{filter basis} of $\mathcal{F}$ if
  \begin{equation*}
    \forall F \in \mathcal{F} \exists F_0 \in \mathcal{F}_0\colon F_0 \subseteq F.
  \end{equation*}
\end{defin}

\begin{lemma}\label{lem:filbas}
  Let $\mathcal{B}$ be a collection of non-empty subsets of a non-empty set $X$. $\mathcal{B}$ is a basis of a filter if and only if
  \begin{equation}\label{eq:filterbasis}
    \forall B_1, B_2 \in \mathcal{B}\exists B_3 \in \mathcal{B}: B_3 \subseteq B_1 \cap B_2.
  \end{equation}
\end{lemma}

\begin{proof}
  Assume $\mathcal{B}$ is the basis of a filter $\mathcal{F}$ on $X$. It holds especially that $\mathcal{B} \subseteq \mathcal{F}$. Let $B_1, B_2 \in \mathcal{B}$. Since these are also filter elements it follows that $B_1 \cap B_2 \in \mathcal{B}$ which proves the claim.

  Now assume that $\mathcal{B}$ fulfills equations \ref{eq:filterbasis} and let $B_1, B_2 \in \mathcal{B}$. Define the filter $\mathcal{F} := \{ F \in \P(X)\setminus\{ \emptyset \}\colon \exists B\in\mathcal{B}\colon B \subseteq F \}$. It is clear that $\emptyset \notin \mathcal{F} \ni X$. Furthermore this filter is closed under the operation of taking supersets by definition. Now suppose that $F_1, F_2 \in \mathcal{F}$. There exist $B_1, B_2 \in \mathcal{B}$ with $B_1 \subseteq F_1$ and $B_2 \subseteq F_2$. Now by equation \ref{eq:filterbasis} it follows that there is $B_3 \in \mathcal{B}$ with $B_3 \subseteq B_1 \cap B_2$ and thus
  \begin{equation*}
    \emptyset \neq B_3 \subseteq B_1 \cap B_2 \subseteq F_1 \cap F_2.
  \end{equation*}
  Hence $F_1 \cap F_2 \in \mathcal{F}$.
\end{proof}

From this point the notation $\mathcal{F}(\mathcal{B}) := \{F \in \P(X)\setminus\{\emptyset\}\colon \exists B\in\mathcal{B}\colon B\subseteq F\}$ represents the filter on $X$ generated by the filter basis $\mathcal{B}$.

\begin{col}\label{cor:filbas}
  Let $X$ be a non-empty set, $A \subseteq X$ and let $\mathcal{F}$ be a filter on $X$. If $A \cap F \neq \emptyset$ for all $F\in\mathcal{F}$ then the set $\mathcal{A} := \{A \cap F\colon F\in \mathcal{F}\}$ forms a filter basis of a filter which is finer than $\mathcal{F}$.
\end{col}

\begin{proof}
  Let $F_1, F_2 \in \mathcal{F}$ and define $A_1 := F_1 \cap A$ and $A_2 := F_2 \cap A$. Now consider
  \begin{equation*}
    A_1 \cap A_2 = (F_1 \cap A) \cap (F_2 \cap A) = F_1 \cap F_2 \cap A = F_3 \cap A \in \mathcal{A}
  \end{equation*}
  where $F_3 = F_1 \cap F_2 \in \mathcal{F}$. Thus $\mathcal{A}$ is a filter basis by lemma \ref{lem:filbas}.

  Now consider the filter $\mathcal{F}(\mathcal{A})$. Since $\mathcal{A}$ is a filter basis of this filter all elements of the form $F \cap A$ for $F\in\mathcal{F}$ are contained in $\mathcal{F}(\mathcal{A})$. Let $F\in\mathcal{F}$ then there is the element $F\cap A$ in $\mathcal{F}(\mathcal{A})$. Since $F\cap A \subseteq F$ it follows that $F\in \mathcal{F}(\mathcal{A})$. 
\end{proof}

\begin{defin} 
  An \textbf{ultrafilter} $\F$ on $X$ is a filter on $X$ with the property that if there is another filter on X called $\F'$ such that $\F \subseteq \F'$ it follows that $\F = \F'$.
  If $\bigcap \F = \emptyset$ the ultrafilter is called \textbf{free}.
\end{defin}

\begin{lemma}\label{lem:ufltlemma}
  Let $X$ be a non-empty set and let $\mathcal{F} \in {\rm UFlt(X)}$ then
  \begin{equation*}
    \forall A \subseteq X\colon A \in \mathcal{F} \lor A^c \in \mathcal{F}.
  \end{equation*}
\end{lemma}

\begin{proof}
  Since $A \cap A^c = \emptyset$ there are no two sets $F_1, F_2 \in \mathcal{F}$ with $F_1 \subseteq A$ and $F_2 \subseteq A^c$. This means that for all $F \in \mathcal{F}$ either $F \cap A \neq \emptyset$ or $F \cap A^c \neq \emptyset$. Assume w.l.o.g. that $A \cap F \neq \emptyset$ for all $F \in \mathcal{F}$. From this it follows that $\{F \cap A\colon F\in \mathcal{F}\}$ is a filter basis of a filter $\mathcal{G}$ which is finer than $\mathcal{F}$ by corollary \ref{cor:filbas}. Since $\mathcal{F}$ is an ultrafilter it follows that $\mathcal{F} = \mathcal{G}$ and thus $A \in \mathcal{F}$. \cite[5.12 Satz]{BvQMT}
\end{proof}

\begin{defin}
  Let $X$ be a topological space. A filter $\F$ converges to an element $x$ in $X$ ($\F \to x$) if $\mathcal{N}_x \subseteq \F$. 
  
  Let $Y$ be another topological space and $f: X \to Y$ a continuous function. 
  We call $f(\F)$ the \textbf{image filter} of $\F$ under $f$ which is the filter with filter basis $\{f(F)\colon F \in \F\}$. Another notation for the convegence of a filter is $\lim\limits_{F\to\F}f(F)$ (instead of $f(\F) \to x$).
\end{defin}

Indeed the new definition of convergence extends the old one from \ref{def:convtop}. Let $X$ be a topological space and let $(x_n)_{n\in\N}$ be a sequence in $X$ which converges to $x \in X$. Then there exists a filter on $X$ which converges to $x$ namely $\mathcal{N}_x$. The inverse is of course not true because the new definition of convergence now allows the convergence to the point $\omega_1$ in the above example. And it is also true that if $A \subseteq X$ and $a \in \bar{A}$ then there exists a filter that converges to $a$. This is not true for sequences by example \ref{ex:ordertop-uncount}.

\begin{thm}
  Let $X$ be a topological space and let $A \subseteq X$. Then
  \begin{equation*}
    x \in \bar{A} \: \iff \: \exists \F \in {\rm Flt}(X): A \in \F \: \land \: \F \to x.
  \end{equation*}
\end{thm}

\begin{proof}
  See \cite[5.17 Satz]{BvQMT}.
\end{proof}

\begin{defin}
  The \textbf{Cofinite-Filter} $\F_{CF}$ on an infinite set $X$ is defined as 
  \begin{equation*}
    \F_{CF} := \{ F \in \P(X)\colon \left| F^c \right| < \infty\}.
  \end{equation*}
\end{defin}

\begin{lemma}\label{lem:coffil}
  Let $X$ be an infinite set, then the Cofinite-Filter on $X$ is a free filter.
\end{lemma}

\begin{proof}
  It is trivial to see that $X \in \F_{CF}\:$ and $\:\emptyset \notin \F_{CF}$. So let $A, B \in \F_{CF}$ then
  \begin{align*}
    (A \cap B)^c = A^c \cup B^c \Rightarrow \left| (A \cap B)^c \right| = \left | A^c \cup B^c \right| \leq \left| A^c \right| + \left| B^c \right| < \infty
  \end{align*}
  since $A^c$ and $B^c$ are finite.
  Now assume that $A \subseteq B$. It follows that
  \begin{align*}
    A \subseteq B \: \Rightarrow \: B^c \subseteq A^c \: \Rightarrow \left|B^c\right| \leq \left|A^c\right| < \infty.
  \end{align*}
  This proofs that $\F_{CF}$ is a filter on $X$.
  Now assume that $\bigcap \F_{CF}$ is non-emtpy and let $x \in \bigcap \F_{CF}$. Now let $F \in \F_{CF}$. By assumption it follows that $x \in F$ and
  \begin{align*}
    \left|(F\setminus \{x\})^c\right| = \left| F^c \right| + \left| \{x \} \right| = \left| F^c \right| + 1 < \infty
  \end{align*}
  and hence $F\setminus \{x\} \in \F_{CF}$. $\lightning$
\end{proof}

\begin{thm}\label{thm:ulfil}
  Let $X$ be non-empty set. Every filter on X is contained in an ultrafiler on $X$.
\end{thm}

\begin{proof}
  Let $\F$ be a filter on $X$ and let $\Phi$ be the set of all filters on $X$ that are finer than $\F$. This set forms a partial order together with the $\subseteq$ relation.
  If $\Phi_1$ is a totally ordered subset of $\Phi$ then define $\F' := \bigcup\limits_{\mathcal{G} \in \Phi} \mathcal{G}$.
  \textbf{Claim: } $\F'$ is a filter on $X$. The first property is clear because every filter contains $X$ and does not contain $\emptyset$.
  Now let $A, B \in \F'$. This means that there are filters $\mathcal{G}_1$ and $\mathcal{G}_2$ in $\F'$ with $A \in \mathcal{G}_1$ and $B \in \mathcal{G}_2$.
  Since $\mathcal{G}_1 \in \Phi_1$ and $\mathcal{G}_2 \in \Phi_1$ it follows that $\mathcal{G}_1 \subseteq \mathcal{G}_2$ or $\mathcal{G}_2 \subseteq \mathcal{G}_1$. Suppose w.l.o.g. that $\mathcal{G}_1 \subseteq \mathcal{G}_2$.
  It follows that
  \begin{equation*}
    A \in \mathcal{G}_2 \Rightarrow A \cup B \in \mathcal{G}_2 \Rightarrow A \cup B \in \F'.
  \end{equation*}
  Now suppose that $A \in \F'$ and $A \subseteq B \subseteq X$. Since there is an $\mathcal{G} \in \Phi_1$ with $A \in \mathcal{G}$ we can conclude that
  \begin{equation*}
    B \in \mathcal{G} \Rightarrow B \in \F'.
  \end{equation*}
  This proofs the claim. $\F'$ is an upper bound of $\Phi_1$ and thus $\Phi_1$ is inductively sorted. The existence of an ultrafilter on $X$ follows from Zorn's Lemma. \cite[5.12 Satz]{BvQMT}
\end{proof}

\begin{col}\label{col:exfreeuf}
  Let $X$ be a infinite set. There exists a free ultrafilter on $X$.
\end{col}

\begin{proof}
  Follows directly from \ref{lem:coffil} and \ref{thm:ulfil}.
\end{proof}

\begin{lemma}
  Every ultrafilter on a non-empty finite set $X$ converges to a point of $X$.
\end{lemma}

\begin{proof}
  In the case of a discrete set, the ultrafilter converges agains an element if the singleton set containing this element is contained inside of the ultrafilter.
  Assume the cardinality of $X$ is 1. Then the claim follows directly by the definition of the ultrafilter.
  Now assume that the lemma is true for a set with cardinality $n \in \N$ and let $X$ be a set with cardinality $n+1$ and additionally let $\mathcal{F}$ be an ultrafilter on $X$. Write $X = \{1, \ldots, n\} \cup \{n+1\}$. By Theorem \ref{lem:ufltlemma} either $\{n+1\} \in \mathcal{F}$ which ends the proof by the definition of convergence of an ultrafilter or $\{1,\ldots,n\} \in \mathcal{F}$ which ends the proof by the previous assumption.
\end{proof}

\subsection{Topological Algebra}

\begin{defin}
  Let $G$ be a group with group operation $\cdot_G$ and $\T$ a topology on the undelying set of $G$. Additionally define $m\colon G \times G \to G, (g, h) \mapsto g \cdot_G h$ as the multiplication map of $G$ and $\iota\colon G \to G, g \mapsto g^{-1}$ as the inverse map of $G$. Then $G$ is called a \textbf{topological group} if both $m$ and $\iota$ are continous maps with respect to $\T$.
  Let $X$ be a set. An action of $G$ on $X$ is a map $\lambda\colon G\times X \to X, (g, x) \mapsto \lambda(g, x) =: g \anddot x$ if it fulfills the following properties
  \begin{enumerate}
    \item $\forall g, h \in G: h \anddot (g \anddot x) = (h \cdot g) \anddot x$,
    \item For neutral element $e_G \in G$: $\lambda(e, x) = x$ for all $x \in X$.
  \end{enumerate}
  If $X$ is a topological space, then the action of $G$ on $X$ is continuous if $\lambda$ is a continuous map.
\end{defin}

% TODO: explenation why this property is so important and hat it is used for
\begin{defin}
  A topological group $G$ is called \textbf{extremly ameanable} if each every continous action of G on a compact space has a fixed point.
\end{defin}

\subsection{Algebraic Topology}

\textbf{Notation:}
\begin{enumerate}
  \item Let $Y$ be a topological space with the topology $\T_Y$,
  \item Let $X$ be a metric space with the metric $d_X$ which is also a topological space with the topology induced by its metric,
  \item If $Z_1, Z_2$ are topological spaces, then $C(Z_1, Z_2)$ is the set of all continuous maps from $Z_1$ to $Z_2$,
  \item Let $y \in Y$, then $\NB_y$ is the set of all neighbourhoods of the element $y$,
  \item For $\varepsilon \in \R_{>0}$ and $x \in X$, let $B_{\varepsilon}(x) := \{ y\in X \: | \: d_X(x, y) < \varepsilon\}$.
\end{enumerate}

\begin{defin}
  A \textbf{path} $\gamma$ in $Y$ is a continuous map $[0,1] \to Y$ where $[0,1]$ is seen as a subspace of $\R$.
\end{defin}

\begin{defin}
  Let $\gamma_1, \gamma_2 \in C([0, 1], Y)$. A \textbf{free homotopy} $H\colon [0,1]^2 \to Y$ between $\gamma_1$ and $\gamma_2$ is a continuous map such that
  \begin{equation*}
    \forall s \in [0,1]\colon H(s, 0) = \gamma_1(s) \: \land \: H(s, 1) = \gamma_2(s)
  \end{equation*}

  Now let $\gamma_1, \gamma_2 \in C([0, 1], Y)$ such that $\gamma_1(0) = \gamma_2(0)$ and $\gamma_1(1) = \gamma_2(1)$ i.e. the two paths start and end at the same point.
  The homotopy $H \in C([0,1]^2, Y)$ is called \textbf{endpoint preserving} if it is a free homotopy between $\gamma_1$ and $\gamma_2$ and it has the following property:
  \begin{equation*}
    \forall t\in[0,1]\forall i\in \{1,2\}\colon H(0, t) = \gamma_i(0) \: \land \: H(1, t) = \gamma_i(1).
  \end{equation*}

  If a free homotopy $H$ between the two paths $\gamma_1$ and $\gamma_2$ exists they are called \textbf{freely homotopic} $\gamma_1 \overset{\cdot}{\sim}_H \gamma_2$.
  If the two paths $\gamma_1$ and $\gamma_2$ have the same start and end point and there exists an endpoint preserving homotopy $H$ between them then they are called 
  \textbf{homotopic} $\gamma_1 \sim_H \gamma_2$. If a path is (freely) homotopic to a constant path ($[0,1] \to Y, t \mapsto y \in Y$) then it is called \textbf{null-homotopic}.

  Lastly define the homotopy relation $\sim$ on $C([0,1], Y)$ where 
  \begin{equation*}
    \forall\gamma_1, \gamma_2 \in C([0,1], Y)\colon \gamma_1 \sim \gamma_2 \iff \exists H \text{ homotopy between } \gamma_1 \text{ and } \gamma_2.
  \end{equation*}
\end{defin}

In this thesis all homotopies are endpoint preserving if it is not explicitly stated that they should be free homotopies.

\begin{defin} \label{def:connectedness}
  \begin{itemize}
    \item[] % this has to be here because of the newline after the definition label
    \item $Y$ \textbf{connected} $\;\longeq\; \forall A,B \in \T_Y\colon (A \cap B = \emptyset \: \land \: Y = A \cup B) \Rightarrow (A = \emptyset \: \lor \: B = \emptyset)$
    \item $Y$ \textbf{path-connected} $\;\longeq\; \forall x,y \in Y \exists\gamma \in C([0,1], Y)\colon \gamma(0) = x \: \land \: \gamma(1) = y$
  \end{itemize}
\end{defin}

\begin{lemma} \label{lem:homotopy-equivalence}
  The homotopy relation $\sim$ of paths is an equivalence relation on the set $C([0,1],Y)$.
\end{lemma}

\begin{proof}
  Let $\gamma_1, \gamma_2, \gamma_3 \in C([0,1],Y)$ with $\gamma_1(0) = \gamma_2(0) = \gamma_3(0)$ and $\gamma_1(1) = \gamma_2(1) = \gamma_3(1)$.
  
  \textit{Reflexivity:}
  Consider the homotopy $H\colon [0,1]^2 \to Y, \: (s, t) \mapsto \gamma_1(s)$. 
  
  It holds that for all $s\in [0,1]\colon H(s, 0) = \gamma_1(s),\: H(s, 1) = \gamma_1(s)$. And thus $\gamma_1 \sim_H \gamma_1 \Rightarrow \gamma_1 \sim \gamma_1$.

  \textit{Symmetry:} 
  Assume that $\gamma_1 \sim \gamma_2$. This means there exists a homotopy $H$ such that $\gamma_1 \sim_H \gamma_2$. 
  Now define the homotopy $F\colon [0,1]^2 \to Y, \: (s, t) \mapsto H(s, 1-t)$. 
  
  From this definition it follows that for all $s \in [0,1]:$
  \begin{align*}
    F(s, 0) &= H(s, 1 - 0) = \gamma_2(s) \\
    F(s, 1) &= H(s, 1 - 1) = \gamma_1(s)
  \end{align*}
  and hence $\gamma_2 \sim_F \gamma_1 \Rightarrow \gamma_2 \sim \gamma_1$.

  \textit{Transitivity:}
  Assume that $\gamma_1 \sim \gamma_2$ and $\gamma_2 \sim \gamma_3$. Let $H$ be a homotopy between $\gamma_1$ and $\gamma_2$ and let $F$ be a homotopy between $\gamma_2$ and $\gamma_3$. 
  Define the homotopy
  \begin{equation*} 
    G: [0,1]^2 \to Y, \: (s,t) \mapsto \begin{cases}
      H(s, 2t), &t \in [0, {1 \over 2}], \\
      F(s, 2t - 1), &t \in [{1 \over 2}, 1].
    \end{cases}
  \end{equation*}

  Then the following equations hold
  \begin{align*}
    G(s,0) &= H(s,0) = \gamma_1(s), \\
    G\left(s, \tfrac{1}{2}\right) &= H(s, 1) = \gamma_2(s) = F(s, 0), \\
    G(s, 1) &= F(s, 1) = \gamma_3(s),
  \end{align*}
  and thus $\gamma_1 \sim_G \gamma_3 \Rightarrow \gamma_1 \sim \gamma_3$.
\end{proof}

\begin{rem} \label{rem:reparam}
  Let $\alpha: [0,1] \to [0,1]$ be a continuous map with $\alpha(0) = 0, \alpha(1) = 1$ and let $\gamma \in C([0,1], Y)$. Then it follows that $\gamma \sim \gamma \circ \alpha$.
\end{rem}

\begin{proof}
  Let $F\colon [0,1]^2 \to Y, (s, t) \mapsto \gamma((1 - t)s + t\alpha(s))$.

  At first let $s \in [0,1]$ fixed and let $\alpha(s) \in [0,1]$ then $(1 - t)s + t\alpha(s)$ is a convex combination of the two points. 
  $[0,1]$ is a convex set and thus for all $t\in[0,1]$ it is true that $(1 - t)s + t\alpha(s) \in [0,1]$. Because of the continuity of the operations $+$ and $\cdot$ in $[0,1]$ 
  and the continuity of $\alpha$ the map $F$ is continuous and the following hold
  \begin{align*}
    F(s, 0) &= \gamma(s), \\
    F(s, 1) &= \gamma(\alpha(s)),
  \end{align*}
  and thus $\gamma \sim_F \gamma \circ \alpha$ which means $\gamma \sim \gamma \circ \alpha$.
\end{proof}

\begin{defin}
  Let $X$ and $Y$ be topological spaces. The \textbf{homotopy class} of $f \in C(X, Y)$ is $[f]_{\sim} = \{g \in C(X, Y)\colon f \simeq g\}$. A continuous function $f \in C(X, Y)$ is calles a \textbf{homotopy equivalence} if there exists a $g \in C(Y, X)$ such that
  \begin{equation*}
    g \circ f \simeq \id_X \: \land \: f \circ g \simeq \id_Y.
  \end{equation*}
  The two spaces $X$ and $Y$ are \textbf{homotopy equivalent} ($X \simeq Y$) if there exists a homotopy equivalence $f \in C(X, Y)$ between them.
\end{defin}

\begin{lemma}
  Let $X$ and $Y$ be topological spaces, then
  \begin{equation*}
    X \cong Y \: \Rightarrow \: X \simeq Y.
  \end{equation*}
\end{lemma}

\begin{proof}
  Let $\psi\colon X \to Y$ be a homeomorphism. Then it follows that $\psi \circ \psi^{-1} \simeq \id_Y$ and $\psi^{–1} \circ \psi \simeq \id_X$.  
\end{proof}
