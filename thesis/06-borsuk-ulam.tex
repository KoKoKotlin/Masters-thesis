\section{The Borsuk-Ulam-Theorem and its Generalization}

The following section describes the important Borsuk-Ulam theorem and a generelization needed to prove the following claim:
\begin{thm}
  If $p \in \mathbb{P}$ and $n, l \in \mathbb{N}$ with $n \geq l$ such that
  \begin{equation*}
    d(p-1) \leq l - n,
  \end{equation*}
  then for every continuous map $f\colon\lVert K_p^{n,l}\rVert \to \R^d$ there is a point in $\lVert K_p^{n,l}\rVert$ whose $\Z_p$-Orbit is mapped to a single point in $\R^d$ by $f$.
\end{thm}
This theorem will be needed later to prove a bound on the chromatic numbers of the graphs $\Gamma(\mu, \P, \varepsilon)$.
