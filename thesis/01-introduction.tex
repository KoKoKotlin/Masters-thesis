\section{Introduction}
The study of amenable groups is an important subfield of topological algebra. Amenability first appeared in the work of John von Neumann \cite{vonNeumann1929} in his investigation of the Banach-Tarski paradox. In von Neumann’s formulation, a group is amenable if there exists a left-invariant mean on the group, meaning a finitely additive probability measure that remains unchanged under translations by group elements.

Later it was discovered that the amenability of a topological group is equivalent to a certain fixed-point property for continuous group actions. Specifically, a topological group is amenable if and only if every affine action of the group on a compact convex set admits a fixed point.

A stronger notion than amenability is extreme amenability. A topological group is said to be extremely amenable if every continuous action on a compact space has a fixed point, without additional restrictions on the nature of the action or the space.

In this thesis, we study a particular class of topological groups, denoted by $L_0(\mu, G)$. The elements of these groups are simple functions that take values in a topological group $G$ and are measurable with respect to a submeasure $\mu$. The structure of such groups depends on two main components: the submeasure $\mu$ and the topological group $G$.

These groups have been extensively studied under various conditions imposed on the submeasure $\mu$ and the group $G$. A central open question in this area concerns the extreme amenability of $L_0(\mu, G)$ when $G$ itself is amenable.

In this thesis, we establish that $L_0(\mu, G)$ is extremely amenable when $\mu$ is a diffuse submeasure and $G$ is an abelian topological group. Our approach links extreme amenability to a graph coloring problem. Specifically, we show that the group is extremely amenable if and only if a certain sequence of graphs admits no uniform finite bound on the number of colors required for a proper coloring. In Section 4, we introduce these graphs and formulate the coloring criterion. In Section 6, we prove a lower bound on the chromatic number of these graphs, using a monotonically increasing function related to their size. Section 5 provides an essential result concerning simplicial complexes associated with these graphs, which is necessary for the proof in Section 6.

Sections 2 and 3 develop the theoretical background required for the main arguments, covering fundamental topics from topology, algebraic topology, topological algebra, simplicial complexes, and homological algebra.

The general definitions of topology presented in Sections 2 and 3 can be found in any standard textbook on topology and algebraic topology. In Section 2, the definitions of general topology are primarily drawn from \cite{MunTop}, with the concepts of filters and key results about them based on \cite{BvQMT}. The definitions related to algebraic topology follow \cite{MunAlTop}, while those concerning topological algebra are taken from \cite{atop2008}.

Section 3 introduces simplicial complexes following \cite{MunAlTop}, and the discussion on singular homology is inspired by Professor Martin Schneider’s lecture Geometry and Topology, as well as \cite{hatcher}. Proofs of more advanced results are cited directly within the respective sections.

The key results in Sections \ref{sec:l0groups}, \ref{sec:borsuk}, and \ref{sec:bounds} are based on the work of Marcin Sabok \cite{sabok2012}.
