\section{Simplicial Homology}

\subsection{Abstract simplicial complexes}

\begin{defin}
    Let $X$ be a non-empty set. Then a collection $\mathcal{K} \subseteq \PowS(X)$ of subsets of $X$ is called a 
    \textbf{simplicial complex} on $X$ if
    \begin{equation*}
        \forall G\in\mathcal{K} \colon F \subseteq G \; \Rightarrow \; F \in \mathcal{K}.
    \end{equation*}
    The elements of the set $V(\mathcal{K}) := \bigcup \mathcal{K}$ are the \textbf{vertecies} of $\mathcal{K}$ and
    elements of $\mathcal{K}$ itself are the \textbf{simplicies}. Given a simplex $\sigma \in \mathcal{K}$ then the sets
    $\; \sigma \setminus \{x\} \;$ for $x \in \sigma$ are called the \textbf{faces} of $\sigma$.
    Let $Y$ be another non-empty set and $\mathcal{L}$ be a simplicial complex on $Y$. A map $f: V(\mathcal{K}) \to V(\mathcal{L})$ is called 
    a \textbf{simplicial} map if
    \begin{equation*}
        \forall \sigma \in \mathcal{K}\colon f(\sigma) \in \mathcal{L}.
    \end{equation*}
    If $f$ is bijective it is an \textbf{combinatorial isomorphism}.
\end{defin}

% Simple example of a simplicial complex

\begin{defin}
    Let $X$ be a set, $H$ be a group and $\mathcal{K}$ be a simplicial complex on $X$. If there is an 
    group action $\lambda \colon H \times X \to X$ of $H$ on $X$ then
    the complex $\mathcal{K}$ is an \textbf{$H$-complex} if the map
    \begin{equation*}
        \lambda_h \colon \mathcal{K} \to \mathcal{K}, \: \sigma \mapsto \{ \lambda(h, \tau)\colon \tau \in \sigma \}
    \end{equation*}
    is a simplicial map for all $h \in H$.
\end{defin}

\begin{defin}
    A poset (partially ordered set) is a pair $(P, \preceq)$ where 
    $P$ is a non-empty set and 
    $\preceq$ is a binary relation on $P$ which has the following properties:
    \begin{enumerate}
        \item \textit{Reflexivity}: $\forall x \in P\colon x \preceq x$,
        \item \textit{Antisymmetry}: $\forall x, y \in P\colon (x \preceq y \: \land \: y \preceq x) \Rightarrow (x = y)$,
        \item \textit{Transitivity}: $\forall x, y, z \in P\colon (x \preceq y \: \land \: y \preceq z) \Rightarrow (x \preceq z)$.
    \end{enumerate}
    A chain in $P$ is a subset $C \subseteq P$ which is totally ordered, i.e.
    \begin{equation*}
        \forall x, y \in C\colon x \preceq y \: \lor \: y \preceq x.
    \end{equation*}
\end{defin}

\begin{ex}
    If $X$ is a non-empty set then the pair $(\PowS(X), \subseteq)$ forms a poset.
\end{ex}

The last example establishes the following definition:
If $X$ is a non-empty family of sets then $P(X) := (X, \subseteq)$ is the poset generated by $X$. The elements of $X$
are the elements of the poset and the binary relation of set inclusion is the partial order.

\begin{defin}
    Let $\mathcal{K}$ be an abstract simplicial complex. The \textbf{barycentric subdivision} $\sd(\mathcal{K})$ of $\mathcal{K}$
    is the abstract simplicial complex with $\mathcal{K}$ as the set of vertecies and all chains in $P(\mathcal{K})$ as simplicies.
\end{defin}

\begin{thm}
    If $\mathcal{K}$ is a abstract simplicial complex, then $\sd(\mathcal{K})$ is a well-defined abstract simplicial complex.
\end{thm}

\begin{proof}
    Let $\sigma \in \sd(\mathcal{K})$. Now let $\tau \subseteq \sigma$. Since $\sigma$ was totally ordered $\tau$ is also totally ordered.
    This means that $\tau \in \mathcal{K}$.
\end{proof}

% TODO: example of barycentric subdivision of the triangle simplex 1,2,3

\subsection{Geometric simplicial complexes}

In the following, we will discuss another class of simplicial complexes. 
This time, they will be subspaces of the $n$-dimensional euclidian space.
These complexes will be polyhedra which means that they are constructed by "gluing" 
together the points, straight line segments, polyhedra and their higher dimensional generalizations.
There will be a connection to the aformentioned abstract simplical complexes namely the geometric realization
of an abstract simplicial complex. It takes the abstract complex and maps it into the euclidian space realizing it as a geometric complex.
It will be clear that these two types of objects internally encode the same combinatorical structure. 

\begin{defin}
    Let $n,k \in \N$ and let $x_0, x_1, \ldots, x_k \in \R^n$. The vectors $x_i$ are called \textbf{linearly independent} if it holds that 
    \begin{equation*}
        \sum\limits_{i=0}^k \alpha_i x_i = 0 \; \iff \; \alpha_0 = \alpha_1 = \ldots = \alpha_k = 0,
    \end{equation*} with $\alpha_j \in \R$ and $j = 0, 1, \ldots, k$.
    The vectors are called \textbf{affinely independent} if
    \begin{equation*}
        \sum\limits_{i=0}^k \alpha_i = 0 \; \land \; \sum\limits_{i=0}^k \alpha_i x_i = 0 \; \Rightarrow \; \alpha_0 = \alpha_1 = \ldots = \alpha_k = 0, 
    \end{equation*} with $\alpha_j \in \R$ and $j = 0, 1, \ldots, k$.
\end{defin}

\begin{defin}
    Let $n,k \in \N$ and let $x_0, x_1, \ldots, x_k \in \R^n$. The convex hull of the vectors $x_i$ is
    \begin{equation*}
        {\rm co}(x_0, x_1, \ldots, x_k) := \{ t_0 x_0 + t_1 x_1 + \ldots + t_k x_k \colon t_i \in \R \; \land \; \sum\limits_{i=0}^k t_i = 0\}.
    \end{equation*}
\end{defin}

Affine independence of the vectors makes sure that the convex hull of vectors is not degenerated.
For example it is expected that the convex hull of three vectors in $\R^2$ is a triangle. But in the degenerate case that the points are all the same
or the points lie in a straight line, this is not true. When it is assumed that the points are affinely independent, these cases are excluded. 