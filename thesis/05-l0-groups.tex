\section{$L_0$-Groups, connection between extrem ameanability and chromatic numbers}
%TODO: cite paper of friedrich martin with the definitions of raikov and submeasures
In this chapter, we want to establish a connection between the extrem ameanability of a $L_0$-Group and the boundedness of the chromatic numbers of a sequence of graphs.
These graphs will be constructed using the structure of the $L_0$-groups.

\begin{defin}
  Let $\mathcal{A}$ be a boolean algebra and $\mu\colon \mathcal{B} \to [0, \infty)$. The map $\mu$ is called a \textbf{submeasure} if
  \begin{enumerate}
    \item $\mu(0) = 0$,
    \item $\forall A, B \in \mathcal{A}\colon A \leq B \Rightarrow \mu(A) \leq \mu(B)$,
    \item $\forall A, B \in \mathcal{A}\colon \mu(A \lor B) \leq \mu(A) + \mu(B)$.
  \end{enumerate}
\end{defin}

From here a submeasure on a set $X$ is a submeasure on the power set algebra on $X$.

\begin{defin}
  Let $X$ be a set. Then ${\rm PartFin}(X)$ is the set of all finite partitions of $X$, which means:
  \begin{equation*}
    {\rm PartFin}(X) := \left\{ \mathcal{A} \subseteq \P(X)\colon (\forall A,B \in \mathcal{A}\colon A \cap B = \emptyset) \: \land \: \bigcup\mathcal{A} = X \: \land \: \left| \mathcal{A} \right| < \infty \right\}.
  \end{equation*}
  Let $\P_1, \P_2 \in {\rm PartFin}(X)$. $\P_1$ is said to \textbf{refine} $\P_2$ ($\P_2 \preccurlyeq \P_1$) if
  \begin{equation*}
    \forall A \in \P_1\exists B \in \P_2\colon A \subseteq B.
  \end{equation*}
  For every $\P \in {\rm PartFin}(X)$ define the map $\iota_\P\colon X \to \P$ with
  \begin{equation*}
    \iota_\P\colon x \mapsto \begin{cases}
      &A_1, \: \text{if } x \in A_1, \\
      &A_2, \: \text{if } x \in A_2, \\
      &\vdots \\
      &A_n, \: \text{if } x \in A_n,
    \end{cases}
  \end{equation*}
  where $k = \left|\P\right|$.    
  %TODO: is this needed?
  The set ${\rm PartFin}_\mu(X)$ contains all finite partitions of $X$ but with the additional constraint that they have to be measurable with respect to a submeasure $\mu$ on the boolean algebra of subsets of $X$. 
\end{defin}

%TODO: directed set in appendix? or definition above?
\begin{thm}\label{thm:algdirected}
  Let $X$ be a set. Then the set ${\rm PartFin}(X)$ together with the binary operation $\preccurlyeq$ is a directed set.
\end{thm}

% TODO: picture of refinments of the square
\begin{proof}
  Reflexivity and transitivity follow directly from the reflexivity and transitivity of the $\subseteq$ relation.
  Now let $\P_1, \P_2 \in {\rm PartFin}(X)$. Define $\P_3 := \{ P \cap P' \colon P \in \P_1, \:P' \in \P_2 \}$.
  Since for any two sets $A, B$ it holds that $A \cap B \subseteq A$ and $A \cap B \subseteq B$ it follows that
  \begin{equation*}
    \P_1 \preccurlyeq \P_3 \land \P_2 \preccurlyeq \P_3.
  \end{equation*}
\end{proof}

\begin{rem}\label{rem:partfinsze}
  If $X$ is an infinite set then ${\rm PartFin}(X)$ is also infinite.
\end{rem}

\begin{proof}
  Define $\P \subseteq {\rm PartFin}(X)$ as follows
  \begin{equation*}
    \P := \Big\{ \{ \{x\}, \: X\setminus\{x\}\}\colon \: \forall x \in X \Big\}.
  \end{equation*}
  It is now clear that $\left| \P \right| = \left| X \right|$ and since $\P \subseteq {\rm PartFin}(X)$ it follows that 
  \begin{equation*}
    \left| X \right| \leq \left| {\rm PartFin}(X) \right|.
  \end{equation*}
\end{proof}

\begin{defin}
  Let $X$ and $Y$ be sets. A step function $f\colon X \to Y$ with finite range induces a finite partion on X called $\P_f$ in the following way
  \begin{equation*}
    \P_f := \{f^{-1}(\{y\})\colon \forall y \in Y\}.
  \end{equation*}
  This is a well-defined finite partion because of the assumption that $f$ has finite range.
\end{defin}

\begin{defin}
  Let $X$ be a set and let $\mu\colon \mathcal{B} \to [0, \infty)$ be a submeasure on $X$ with on the boolean algebra $\mathcal{B} \subseteq \P$. Then the set
  \begin{equation*}
    S(\mu, G) := \left\{ f: X \to G\colon f \:\:\mu-\text{measureable} \: \land \: \left| f(X) \right| < \infty \right\}
  \end{equation*}
  for a topological group $G$ is the set of all measurable step functions with finite range on $X$ and values on $G$.
\end{defin}

\begin{thm}
  The set $S(\mu, G)$ for a submeasure $\mu$ on a set $X$ and a topological group $G$ is a group with group operation
  \begin{equation*}
    \star\colon S(\mu, G) \times S(\mu, G) \to S(\mu, G), (f, g) \mapsto f \star g
  \end{equation*}
  where $(f\star g)(x) = f(x) \cdot_G g(x)$ for each $x \in X$. If $f$ is an element of $S(\mu, G)$ then define $f^{-1}$ as $f^{-1}(x) = (f(x))^{-1}$ for all $x \in X$. If we put the topology of convergence in submeasure on $S(\mu, G)$ it the group
  becomes a topological group.

  A basis element of the topology of convergence in submeasure is given by
  \begin{equation*}
    V_\varepsilon(f) := \{ h \in S(\mu, G)\colon \mu(\{x \in X\colon h(x) \notin V\cdot_G f(x)\}) < \varepsilon \}
  \end{equation*}
  where $V$ is a neighborhood of the identity on $G$, $f \in S(\mu, G)$ and $\varepsilon \in \R_{>0}$.
\end{thm}

\begin{proof}
  Firstly let $f, g \in S(\mu, G)$. Since $f$ and $g$ induce partitions $\P_f, \P_g$ and the partition $\P = \{A \cap B\colon A \in \P_f, B \in \P_g\}$ is again a measurable partition it is clear that $f \star g \in S(\mu, G)$. The associativity, existence of a neutral element and existence of an inverse element follow from the fact that $\cdot_G$ is a group operation.
  It remains to show that ${\rm mul}\colon S(\mu, G)\times S(\mu, G) \to S(\mu, G), (f, g) \mapsto f\star g$ and ${\rm inv}: S(\mu, G) \to S(\mu, G), f \mapsto f^{-1}$ are continuous. To this end let $V$ be a neighborhood of the identity of $G$, $f \in S(\mu, G)$ and let $\varepsilon \in \R_{>0}$. Consider
  \begin{align*}
    {\rm inv}^{-1}(V_\varepsilon(f)) &= \{h \in S(\mu, G)\colon \mu(\{x \in X\colon (h(x))^{-1} \notin V\cdot f(x)\}) < \varepsilon\} \\
    &= \{h \in S(\mu, G)\colon \mu(\{x \in X\colon h(x) \notin V^{-1}\cdot (f(x))^{-1}\}) < \varepsilon\}
  \end{align*}
  where $V^{-1} := \{g^{-1}\colon g \in V\}$.
  This is a neighborhood of $f^{-1}$ since $V^{-1}$ is also a neighborhood of the identity of $G$. 
  Now consider $M := {\rm mul}^{-1}(V_\varepsilon(f))$ which means that \[M = \{(h, g) \in S(\mu, G) \times S(\mu, G)\colon \mu(\{x \in X\colon (h \star g)(x) \notin V\cdot f(x)\}) < \varepsilon\}.\]
  This set is open since
  \begin{align*}
    \pi_1(M) &= \{h \in S(\mu, G)\colon \exists g\in S(\mu, G)\colon \mu(\{x \in X\colon (h \star g)(x) \notin V\cdot f(x)\}) < \varepsilon\} \\
    &= \{h \in S(\mu, G)\colon \exists g\in S(\mu, G)\colon \mu(\{x \in X\colon h(x) \cdot g(x) \notin V\cdot f(x)\}) < \varepsilon\} \\
    &= \{h \in S(\mu, G)\colon \exists g\in S(\mu, G)\colon \mu(\{x \in X\colon h(x) \notin V\cdot f(x) \cdot (g(x))^{-1}\}) < \varepsilon\} \\
    &= \{h \in S(\mu, G)\colon \exists \tilde{f}\in S(\mu, G)\colon \mu(\{x \in X\colon h(x) \notin V\cdot \tilde{f}(x)\}) < \varepsilon\} \\
    &= \bigcup\limits_{\tilde{f} \in S(\mu, G)}\{h \in S(\mu, G)\colon \mu(\{x \in X\colon h(x) \notin V\cdot \tilde{f}(x)\}) < \varepsilon\}.
  \end{align*}
  $\pi_1$ is the projection to the first component. By symmetry reasons this also works for the second component.
\end{proof}

This set of simple functions is dense in the set $L_0(\mu, G)$. But since in general $L_0(\mu, G)$ is not a metric space, the standard metric notion of completion does not work. Instead it is true that the Raikov completion $\widehat{S(\mu, G)} = L_0(\mu, G)$. Look at \cite{} for further information of this completion. This fact about the simple functions is needed because of the following theorem:
\begin{thm}
  
\end{thm}

In the following that symbols $\bar{1}$ represents the constant function which sends each element to 1 in the set $S(\mu, \Z)$. 

\begin{lemma}\label{lem:1}
  Let $(X, \mathcal{B})$ be a measurable space and let $\mu$ be a submeasure on $X$. Additionally, let $\P \in {\rm PartFin}(X)$ and $\varepsilon \in \R_{>0}$ then for $f,g \in \Z^\P$ with $f - g \in \bar{1} + V_\varepsilon$ it follows that $f$ and $g$ are connected in $\Gamma_\varepsilon^\P(\mu)$ by an edge.
\end{lemma}

\begin{proof}
  \begin{align*}
    f - g \in \bar{1} + V_\varepsilon   &\iff \mu(\{x \in X\colon (f-g)(x) \neq 1\}) < \varepsilon \\
                                      &\iff \mu(\{x \in X\colon f(x) \neq g(x) + 1\}) < \varepsilon \\
                                      &\iff \mu(\bigcup \{A \in \P \colon f(A) \neq g(A) + 1\}) < \varepsilon
  \end{align*}
\end{proof}

\begin{thm}\label{thm:colve}
  Let $X$ be an infinite set, $\mu\colon \mathcal{B} \to [0, \infty]$ a submeasure on $X$ where $\mathcal{B}$ is a $\sigma$-algebra and let $G$ be an abelian group. Then the following are equivalent:
  \begin{enumerate}
    \item $L_0(\mu, G)$ is extremely ameanable,
    \item $\forall \varepsilon > 0$ there is no finite bound on $\chi(\Gamma_{\varepsilon}^{\mathcal{P}}(\mu))$, where $\mathcal{P} \in {\rm PartFin}(X)$.
  \end{enumerate}
\end{thm}

%TODO: define S(\mu, G) and the property of inheritance of ameanability through the raikov completion
%TODO: syndetic set definition
%TODO: correctly cite pestov
\begin{proof}
  Assume there exists a finite bound $d \in \N$ for the chromatic number for some $\varepsilon \in \R_{>0}$, i.e.
  \begin{equation*}
    \forall \P \in {\rm PartFin}(X)\colon \chi(\Gamma_\varepsilon^{\P}(\mu)) \leq d. 
  \end{equation*}
  Given this assumption we claim that the the group $L_0(\mu, \Z)$ is not extremely ameanable. To this end choose a coloring of $\Gamma_{\varepsilon}^{\P}(\mu)$ called $c_{\P}\colon \Z^{\P} \to \{1, \ldots, d\}$ for every $\P \in {\rm PartFin}(X)$. Consider the family of sets
  \begin{equation*}
    \mathcal{A} := \{\{ \P \in {\rm PartFin}(X)\colon \P_0 \preccurlyeq \P\}\colon \P_0 \in {\rm PartFin}(X)\}.
  \end{equation*} 
  This is a well-defined filter basis by Theorem \ref{thm:algdirected} and Lemma \ref{lem:filbas}. From Theorem \ref{thm:ulfil} it follows that there is an ultrafilter $\mathcal{U}$ containing the filter $\mathcal{F}(\mathcal{A})$.
 Let $\P_f$ be the finite partion of $X$ induced by $f \in S(\mu, \Z)$. Define $f_\P \in \Z^\P$ for all finite partitions $\P$ refining the partion $\P_f$ as
  \begin{equation*}
    f_\P(A) = k \iff \exists B \in f^{-1}(\{k\})\colon A \subseteq B. 
  \end{equation*}
  This functions is well-defined because of the definition of refinment of partitions and thus also $f \in S(\mu, \Z)$.
  Now define a coloring $c: S(\mu, G) \to \{1, \ldots, d\}$ of $S(\mu, G)$ as the limit over the ultrafilter
  \begin{equation*}
    c(f) := \lim\limits_{\P \to \mathcal{U}} c_{\P}(f_\P).
  \end{equation*}
  Let $X_i := \{f \in S(\mu, \Z)\colon c(f) = i\}$ for each $i = 1, \ldots, d$ which is a cover of $S(\mu, \Z)$ and let $f, g \in \Z^\P$. By Lemma \ref{lem:1} it follows that $c_\P(f) \neq c_\P(g)$ for $f - g \in \bar{1} + V_\varepsilon$.
  In the limit of the ultrafilter $\mathcal{U}$ we get $(\bar{1} + V_\varepsilon) \cap (X_i - X_i)$ for each $i = 1, \ldots, d$ and thus $X_i - X_i$ is not dense at $\bar{1}$. By Pestovs characterization \cite[Theorem 3.4.9]{PestovDyn} of extrem amenability it follows that $S(\mu, \Z)$ is not extremely ameanable. Hence, $L_0(\mu, \Z)$ is not extremly ameanable by Theorem \ref{}. 

  For the second implication suppose that $S(\mu, G)$ is not extremely ameanable. Let $d \in \N$ and let $S_1, \ldots, S_d$ be a finite covering of $S(\mu, G)$ such that there is an $i \in \{1, \ldots, d\}$ with $G \neq \overline{S_i - S_i}$. The existence of such an finite covering with an element that is not everywhere dense follows from Pestovs characterization of extreme ameanability \cite[Theorem 3.4.9]{PestovDyn}.
  Now let $S := S_i$ and let $f \in S(\mu, G) \setminus \overline{S - S}$. It follows that there exist $W \in \mathcal{N}_G(e_G)$ and $\varepsilon \in \R_{>0}$ such that
  \begin{equation*}
    (f + W_\varepsilon) \cap (S - S) = \emptyset
  \end{equation*}
  for $W_\varepsilon := \{h \in S(\mu, G)\colon \mu(\{x\in X\colon h(x) \notin W\}) < \varepsilon\}$ since $f$ is not in the closure of $S - S$. 
  Now let $V_\varepsilon := \{h \in S(\mu, G)\colon \mu(\{x\in X\colon h(x) \neq e_G\}) < \varepsilon\}$. Since $e_G \in W$ it follows that $V_\varepsilon \subseteq W_\varepsilon$ and hence
  \begin{equation*}
    (f + V_\varepsilon) \cap (\overline{S - S}) = \emptyset.
  \end{equation*}
  Define the partition $\P_f$ induced by $f$.
  Now let $\P \in {\rm PartFin}(X)$ with $\P_f \preccurlyeq \P$ and let $k \in \Z^\P$.
  Define
  \begin{equation*}
    g_k\colon X \to G, \: g_k(x) \mapsto k(\iota_\P(x))f(x)
  \end{equation*}
  which is an element of $S(\mu, G)$ and define the mapping $c\colon \Z^\P \to \{1,\ldots, d\}$ as
  \begin{equation*}
    c\colon k \mapsto i \iff g_k \in S_i.
  \end{equation*}

  \textbf{Claim:} The mapping $c$ is a coloring of the graph $\Gamma_\varepsilon^\P(\mu)$. To this end suppose there are two nodes $k, l \in \Z^\P$ which are connected and have the same color $i \in \{1, \ldots, d\}$. Let $B := \bigcup\{A\in\P\colon k(A) \neq l(A) + 1\}$. It holds that $\mu(B) \leq \varepsilon$ since $k$ and $l$ are connected in $\Gamma_\varepsilon^\P(\mu)$. For $x \in X\setminus B$ it holds that
  \begin{equation*}
    g_k(x) - g_l(x) = k(\iota_\P(x))f(x) - l(\iota_\P(x))f(x) = (k(\iota_\P(x)) - l(\iota_\P(x)) f(x) = f(x)
  \end{equation*}
  since $k(A) = l(A) + 1$ for $x \notin B$. It follows that
  \begin{equation*}
    \mu(\{x\in X\colon g_k(x) - g_l(x) \neq f(x)\}) < \varepsilon.
  \end{equation*}
  which means $g_k - g_l \in f + V_\varepsilon$. Furthermore we have $g_k, g_l \in S_i$ becasue of the definition of $c$. Hence $g_k - g_k \in (f + V_\varepsilon) \cap (S_i - S_i) \lightning$.
  This proofs the claim that $c$ is a coloring of $\Gamma_\varepsilon^\P(\mu)$ with $d$ colors.
\end{proof}
